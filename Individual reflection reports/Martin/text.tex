Requirements 
This reflective report is not directly part of the business care report and will therefore NOT be submitted to the company. It must be written individually and should not exceed two pages. You are thus required to be brief, precise and select your words carefully.

In this report you should critically reflect upon your experiences from the project and relate them to some of the course literature. You may chose either your experiences as a starting point, or concepts from literature. You can use your experiences to reflect, discuss, analyze, or compare the literature, or vice versa.

How these experiences acquired from the IT design project relates to the literature studied in the course (e.g., does your personal experience confirm or refute what is written in the literature?).

You may choose any experience from your project that you find particularly interesting. This can be, for example, a specific challenge you struggled with, a particular incident, a condition at the company (e.g., political organizational issues), an interesting lesson that you have learned, or something that surprised you in your project or the literature.

You can choose any part of the literature, including the book (Bødker, et al., 2006; Osterwalder et al., 2009) as well as the articles in the compendium.

Examples of topics you can discuss:
Impact of iCT on companies, and relations between ICT, organization, and qualifications when designing coherent vision for change (Bødker,  et al., 2006, ch 2.1)
Participation and mutual learning processes between IT designers and management/users (Bødker,  et al., 2006, 2.2)
Say-do problems and the principle of firsthand experiences with work practices (Bødker,  et al., 2006, ch 2.3 \& 6; Blomberg at al., 1993)
The anchoring of visions and challenging your assumptions and hypotheses (Bødker,  et al., 2006; Turban et al., 2010, ch 13)
Project management (Bødker,  et al., 2006, ch 1.4 \& 4; Turban et al., 2010, ch 2, 13 (14-15))
Strategic alignment and the translations from environment and business strategy to work domains that are important to analyze (Bødker,  et al., 2006, ch 5; Turban et al., 2010, ch 13)
Analyzing situation and ambitions (Bødker,  et al., 2006, figure 4.2, 5.1, 6.1, and 7.1)
Discuss one or more of the models, techniques and representation tools that proved to be important in your design project (Bødker,  et al., 2006; Turban, et al., 2010)
Generic standard systems vs. developing new customized systems (Bødker, et al., 2006; Pollock, Williams \& Procter, 2003; Turban et al., 2010 ch 2-3)
Sociotechnical approaches vs. business process reengineering (Hammer, 1990; Berg, 2001; Bødker, et al., 2006; Turban et al., 2010, ch 14 p. 539)
Organizational learning, knowledge equisition, change mangement (Bødker,  et al., 2006; Lyytinen \& Robey, 1999; Lyytinen \& Robey, 1999; Dreyfus, 1997; Turban et al. 2010, ch 14)
Work processes (Bødker,  et al., 2006; Randall, et al., 2007; Suchman, 1983; Orlikowski \& Gash, 1994).
The relationship between ethnography and design (Blomberg, et al., 1993; Randall, et al., 2007. 5-6; Bødker, et al., 2006)
Work processes and artefacts (Randall, et al., 2007)