\section{Ethnographic literature and field methods}
While much of the literature on the course was focused on giving us tools to use or paths to follow, the ethnographic material, for the most part, served a different purpose.
It supplied us with background knowledge of what we were doing and why, gave us terminology to communicate our analyses with, opened our eyes to many different facets of organizational practises, and made us aware of our limited and preconceived knowledge of other cultures or workplaces.
As such, it wasn't directly useful, but I am sure our analyses would have suffered without it.

\subsection{Interviews and observations}
One place where the ethnographic literature proved directly useful, was in the description of and preparation for interviews and observations in Blomberg (1993).

We prepared for the interviews in a very structured manner, listing first the information we knew we wanted, the questions we would ask to get it, and some thoughts on how to ask the questions.
During all this we designed the questions to be open and the interview loose, such that any information we didn't know to ask for had a chance to come through.
I think this is some of the reason why we could get by with as few interviews as we have. 
In most cases, probably due to our analyses beforehand and our insider, we were right about the information we sought, and in the few that we weren't, email and casual communication were sufficient to expand upon our data.

The observations in our case seemed of little use.
As recruitments happen quite rarely, we were unable to observe it 'live', so our gains from observation seemed limited to a general feel of how the departments interacted with each other, what the environment was like.
This, of course, was still useful, but a way off from how important it was presented in the literature.
I agree, though, that observations are necessary in many cases, and that for the sake of learning we should, in this case, do it even if it gains us little.