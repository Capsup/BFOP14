Jakob Ambeck Vase, 21/10/2014

Jeg introducerer mig selv og sætter mig ved et bord for at opleve. Stemningen er lidt underlig, da jeg sidder og kigger på deres arbejde.

Lisbeth skriver meget og kigger mindre.

Annette læser og læser.

Katja læser mest, skriver lidt.

Lisbeth “Annette vi kunne ikke overføre penge til den der canadiske konto?”

“Nej, det ku vi ikke.”

“Så gør vi som vi plejer.”

Mange maskinelle lyde, klik af taster, printer, stoleknirk.

Rummet er aflangt, Lisbeth sidder med ryggen mod den bagerste væg, Annette og Katja deler et bord i midten. Skærmene er vendt væk fra folk som kigger ind. På den ene langvæg er et arkiv af mapper, nogle bag lås men de fleste frit tilgængelige. På den anden langvæg er 4 vinduer med rullegardin trukket op. Annette har 2 skærme, de andre har 1. Kuglepenne, tape, stempler, kaffe, vandkander og glas står mange steder, hulmaskiner, servietter, planter i vinduerne. Papirer er alle vegne. Noget på bordene, noget i standere (som bruges til vigtige papirer normalt.) Skrivebordslamper.

Lisbeth viser mig rundt i deres QHR system, og giver mig også et hurtigt indblik i maconomy og lønsystemet. Hun har en enkelt post-it på skærmen, og har på venstre side en kontrakt hun i løbet af samtalen hiver frem og viser mig noget fra.

Hun snakker lidt om hvad de gjorde i gamle dage i QHR og hvordan det virkede, og hvad der virker i dag.

Jeg spørger om de sender andet end en kontrakt med i “kontrakt-brevet”, og hun svarer ja, der er et medarbejder-retningslinier dokument som bliver sendt med, samt bonus-ordninger og andet.

Nogen gange har nyansatte ønskede initialer.

Jeg nævner Peters forslag med at sende et spørgeskema med i kontrakten, og hun syns det er en god ide.

Jeg siger tak for rundvisningen og skriver videre.

Der er ikke mange personlige ejendele på kontoret.

Lisbeth går ud og snakker med en på gangen, ender med at sige “Nej, så må I vente” eller noget der ligner, og sætter sig ind på kontoret igen med et suk.

Malerier på væggene, loftsbelysningen er omvendte lysstofrør.

Mere skriven og knirken.

Jeg siger tak for tiden og går.