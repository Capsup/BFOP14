\emph{Description:} Create a template for recruitment so recruiters know what information is required. A template also functions as a checklist, ensuring that the writer knows what information to send.

\emph{Pros:} Could reduce the amount of missing information in the initial email to accounting or IT. 
Also leads to a better defined process.

\emph{Cons:} Requires additional work by recruiter. 
If template is too information heavy, some may ignore it.
Valcon employees dislike forms and rigid processes, so they may ignore it.

\subsubsection{Finance}
This solution would save appr. 6.500 DKK in the first year and appr. 8.000 anually afterwards, including initial costs and maintenance.

\subsubsection{Risks}
Should the template get out of date, irrelevant information would have to be sorted through everytime making it take longer time and defeat the purpose of the template. 
It's also a risk that the template won't be used, since it's work waste on implementing the template then and the process remains ineffective.

\subsubsection{Handling the risks}
To deal with the risks, it's important to revise the template every once in a while, every 3 months could be an idea.
To keep the recruiters interested in using the template, it might be relevant to remind them that the template makes the recruiting process smoother and faster, as long as the template is used properly.

\subsubsection{Conclusion} Creating a template for OMT recruitment seems a good idea.
Creating one for Valcon would be a bad idea, as it would take time from management and consultants, but creating one for Hanne might work, as she would be reminded of the information required.