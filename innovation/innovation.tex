In this section we will list the key problems we have found, list our solution propositions, describe them in more detail, and finally, on the basis of a cost-benefit analysis, recommend some of them and discourage others.

\section{Problems and solutions}
\subsection{List of key problems}
\begin{enumerate}
\item IT-department
	\begin{enumerate}
	\item ???
	\end{enumerate}
\end{enumerate}

\subsection{List of solution propositions}
\begin{enumerate}
\item Software that does it all
\item Template for recruitment
\item Change of attitude toward IT
\item Buffer of ready computers
\item Increase IT staff size
\item New HR-system
\item Other small improvements
	\begin{enumerate}
	\item Phone and internet paper in contract letter
	\item Help-program for finding initials and reading templates
	\item "Unkown department" in IT-systems
	\item Mail from OMT when recruitment is assured, but data is not yet available
	\item Further automation of install process
	\end{enumerate}
\end{enumerate}

\subsection{Explanation of solution propositions}
\subsubsection{Software that does it all}
\emph{Description:} This proposal is for a piece of software that can encompass all the requirements of every employee in the company. Lotus Notes had this status once, but many disliked it.

\noindent \emph{Pros:} Only one place for information, easy to manage.

\noindent \emph{Cons:} Very expensive, very hard to accomodate everybody.

\subsubsection{Template for recruitment}
\emph{Description:} When a new recruitment process starts, instead of a mail being sent with the information, a template is used to ensure that no information is missing.

\noindent \emph{Pros:} No information missing means no need to contact employee more than once, saving time.

\noindent \emph{Cons:} Will put more work on recruiter, and Hanne stated that employees at Valcon dislike templates, so they won't use them.

\subsubsection{Change of attitude toward IT}
\emph{Description:} Change the general attitude toward IT. Currently, IT is generally regarded as a department that just solves whatever problem one might have. Changing this attitude to better reflect the level of stress in IT could have benefits.

\noindent \emph{Pros:} If people are more aware of how IT functions, they are more likely to be considerate. We think it would reduce the number of urgent cases and make IT more happy. Doesn't increase workload of consultants or management.

\noindent \emph{Cons:} Hard to implement, hard to measure.

\subsubsection{Buffer of ready computers}
\emph{Description:} Have a few computers installed and ready for use at Valcon and OMT.

\noindent \emph{Pros:} Better response times to urgent recruitments. Useful if an employee's computer breaks at a critical time.

\noindent \emph{Cons:} Needs a little more managing from IT. Computers will need to be updated. Resources not in use.

\subsubsection{Increase IT staff size}
\emph{Description:} Increase the IT staff size to increase throughput.

\noindent \emph{Pros:} Makes IT capable of doing more.

\noindent \emph{Cons:} Expensive, and management increases.

\subsubsection{New HR system}
\emph{Description:} Implement a new HR system for communication between accounting and employers.

Valcon are already looking at new HR systems.

\noindent \emph{Pros:} Process becomes more standardized and information more consistent. Copying data fewer times reduces probability of errors.

\noindent \emph{Cons:} Costly and requires training.

\subsubsection{Other small improvements}
The following propositions are minor, and require little explanation:
\begin{itemize}
	\item Phone and internet paper in contract letter\\
	
			Put a paper in the contract letter sent to new employees, asking for their internet and phone preferences. Give it to IT upon return. Would help reduce missing data for new employees, and save time in IT.

	\item Help-program for finding initials and reading templates\\
	
			Write a small program that can read the template proposed above and enter the information into the IT systems.
			Also able to propose initials for employees, and check whether the exchange mailbox has been used before.
			Would reduce workload for IT and introduce fewer errors, but would require some maintenance and some time initially.
	
	\item "Unkown department" in IT-systems\\
	
			Create a department for unknown departments in the IT systems, possibly both for Valcon and OMT, and create a process for managing it.
			Would remove the department requirement from IT, allowing for quicker recruitment.
			
	\item Mail from OMT when recruitment is assured, but data is not yet available\\
	
			Have OMT write a mail to IT when they know they will hire an employee soon, but don't have the data yet, so IT can plan accordingly.
			Would reduce stress, but increase planning for OMT.
			
	\item Further automation of install process\\
	
			The install process is very time consuming, and is a candidate for automation.
			Looking into whether more of it could be automated could save time.
\end{itemize}

\section{Recommendations}
Based on our cost-benefit analysis (see p.~\pageref{app:cost_benefit_analysis}), we recommend solutions 2, 3, 4 and 6.
Create a template for recruitment for OMT and possibly for Hanne, change the company attitude toward IT, have a buffer of computers ready at both OMT and Valcon, and implement a new HR system.

We would discourage implementing 1, the software that does it all, as it would increase the workload on consultants and management, decreasing work enjoyment and possibly reducing revenue.

Finally, we would advice considering the minor improvements, as they may improve the recruitment process.

\subsection{Other recommendations}
Conduct an analysis of whether the OMT process can be improved so they know further in advance who and when someone is needed, so the number of urgent cases can be reduced.