In this section we will describe the most relevant solution propositions and, on the basis of a cost-benefit analysis, recommend some of them and discourage others.

\section{Solutions}
Please refer to appendix \ref{app:solution_propositions} (p. \pageref{app:solution_propositions}) for a complete list of solution propositions.

\subsection{Template for recruitment}
\emph{Description:} Create a template for recruitment so recruiters know what information is required. A template also functions as a checklist, ensuring that the writer knows what information to send.

\emph{Pros:} Could reduce the amount of missing information in the initial email to accounting or IT. 
Also leads to a better defined process.

\emph{Cons:} Requires additional work by recruiter. 
If template is too information heavy, some may ignore it.
Valcon employees dislike forms and rigid processes, so they may ignore it.

\emph{Conclusion:} Creating a template for OMT recruitment seems a good idea.
Creating one for Valcon seems unlikely to work, but might work for Hanne.

\section{Recommendations}
Based on our cost-benefit analysis (see p.~\pageref{app:cost_benefit_analysis}), we recommend solutions 2, 3, 4 and 6.
\begin{itemize}

\item{Create a template for recruitment for OMT and possibly for Hanne.}
\item{Change the company attitude toward IT}
\item{Have a buffer of computers ready at both OMT and Valcon}
\item{Implement a new HR system.}
\end{itemize}

We would discourage implementing 1, the software that does it all, as it would increase the workload on consultants and management, decrease work enjoyment due to stricter rules and possibly reduce revenue.

Finally, we would advice considering the minor improvements, as they may improve the recruitment process.

\subsection{Other recommendations}
Conduct an analysis of whether the OMT process can be improved so they know further in advance who and when someone is needed, so the number of urgent cases can be reduced.