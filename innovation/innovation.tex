In this section we will describe the most relevant solution propositions and, on the basis of a cost-benefit analysis, document the impact on the company as a whole.

\section{Vision for overall change}
\todo{do}


\section{Solutions}
\todo{Description, pros/cons, cost/benefit for hver}
Please refer to appendix \ref{app:solution_propositions} (p. \pageref{app:solution_propositions}) for a complete list of solution propositions. Each solutions is listed with a verdict of its suitability, sorting out the most improbable.
The most relevant solutions are subject to cost/benefit analysis in the following sections.

\subsection{Template for recruitment}
\emph{Description:} Create a template for recruitment so recruiters know what information is required. A template also functions as a checklist, ensuring that the writer knows what information to send.

\emph{Pros:} Could reduce the amount of missing information in the initial email to accounting or IT. 
Also leads to a better defined process.

\emph{Cons:} Requires additional work by recruiter. 
If template is too information heavy, some may ignore it.
Valcon employees dislike forms and rigid processes, so they may ignore it.

\subsubsection{Finance}

\subsubsection{Risks}
Should the template get out of date, irrelevant information would have to be sorted through everytime making it take longer time and defeat the purpose of the template. 
It's also a risk that the template won't be used, since it's work waste on implementing the template then and the process remains ineffective.

\subsubsection{Handling the risks}
To deal with the risks, it's important to revise the template every once in a while, every 3 months could be an idea.
To keep the recruiters interested in using the template, it might be relevant to remind them that the template makes the recruiting process smoother and faster, as long as the template is used properly.

\subsubsection{Conclusion} Creating a template for OMT recruitment seems a good idea.
Creating one for Valcon would be a bad idea, as it would take time from management and consultants, but creating one for Hanne might work, as she would be reminded of the information required.

\subsection{Change of attitude toward IT}
\emph{Description:} Change the company attitude toward IT, from the current "they are there to fix my problems" to "they are there if I really need them".
We do not know how to implement this, but as Valcon is a consulting company, we assume they have a strategy for this case.

\emph{Pros:} May reduce the number of edge cases, recruiters would be more likely to send recruitment information in early or give warning that a quick recruitment is coming up.
Doesn't require extra time from recruiters.

\emph{Cons:} It would be hard to measure the success of the change - when is an attitude changed? - but measuring the result of the change could be done.

\subsubsection{Finance} For benefit, measure current number of edge cases and their average time cost.
Make a guess as to how many of these would be avoided with the change (15\%?).
Compute hours gained and multiply with wages for IT.
As for cost, 8-10 hours of work for a consultant might be enough to implement this.

\subsubsection{Risks}
It might not be possible to actually change the attitude that employees have towards IT already. Not only that, but putting focus on that incorrectly, could make the attitude towards IT even worse.

\subsubsection{Handling risks}
Should the risks mentioned above be realised for this solution, a way to handle it would be to launch another campaign, effectively trying again. 

\subsubsection{Conclusion} 

\subsection{Buffer of computers}
\emph{Description:} Keep 2-3 pre-installed computers ready for use at OMT and at Valcon, for emergencies or very quick recruitments.

\emph{Pros:} Would reduce response times to urgent recruitments.
Also useful if an employee's computer breaks, to reduce downtime.

\emph{Cons:} Needs a little more managing from IT, as
computers will need to be kept updated and ready.
As all computers are not in use, this creates a small overhead.

\subsubsection{Finance} Free. Benefit in time gained. Resources lying dormant.

\subsubsection{Risks}
Maintaining a buffer of computers could result in the computers getting outdated in case the buffer is too big compared to the amount of new hires.

\subsubsection{Handling risks}
Reducing the amount of computers kept in the buffer will reduce the risk of them getting outdated

\subsubsection{Conclusion}

\subsection{New HR system}
\emph{Description:} Acquire a new HR system to facilitate communication between accounting, IT, and recruiters.
Valcon is already looking at HR systems.

\emph{Pros:} Recruitment process becomes more standardized, as it always follows the same path.
Information becomes more consistent, as personal spreadsheets are less needed. 
This also safeguards against employees leaving.
Errors are less likely, as number of manual entries of information is reduced.

\emph{Cons:} Costly to implement, requires training of accounting, IT, and recruiter employees.
Also requires maintenance.

\subsubsection{Finance} +: Try to put a price on errors, count how many of them occur, guess how many will be eliminated, sum their costs.
Price of current HR system (Lotus Notes).

-: Price of new HR system.
Price of data transfer.
Price of training.
Price of maintenance.

\subsubsection{Risks}
If the new HR system isn't compatible with current systems and processes, it could convolute the process even further. 
If the new HR system isn't optimal and remains unused even after being introduced, it would just be yet another system having to be maintained and introduces additional completely in the process.

\subsubsection{Handling risks}
A major part in handling the risks associated with this solution is analysing the new HR system and making it absolutely sure that it's compatible with the current systems and the entire process. 
Introducing the new system incrementally could also reduce the risks associated as it would be possible to receive feedback early and change the system accordingly. 
Training the employees in the new system is also necessary to reduce the risks involved, as employees who doesn't know the system entirely, may not be able to use it properly.

\subsubsection{Conclusion} This solution is very necessary, as it is the only way information becomes more consistent.

\subsection{Other small improvements}

Write! \todo{The small improvements need including in the report.}


\section{Implementation strategy}
To implement the ideas explained in this section, we suggest that Valcon takes a 'plunge' approach to the implementation, meaning the entire vision of the overall change is implemented all at once.
We see no reason to incrementally introduce the solutions to the company, since none of them are mutually exclusive and can easily be introduced at the same time since none of the solutions carry much risk and it reduces the amount of time necessary for the employees to get accustomed to the new processes.

\section{Other recommendations}
Need rewriting \todo{Is this section good? Is it relevant?}

\subsection{Analyze OMT process}
Conduct an analysis of whether the OMT process can be improved so they know further in advance who and when someone is needed, so the number of urgent cases can be reduced.

\subsection{Further automation of install process}
The install process is very time consuming, and is a candidate for automation.
Looking into whether more of it could be automated could save time.

