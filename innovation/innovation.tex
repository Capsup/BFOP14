In this section we will describe the most relevant solution propositions and, on the basis of a cost-benefit analysis, recommend some of them and discourage others.

\section{Solutions}
Please refer to appendix \ref{app:solution_propositions} (p. \pageref{app:solution_propositions}) for a complete list of solution propositions.

\subsection{Template for recruitment}
\emph{Description:} Create a template for recruitment so recruiters know what information is required. A template also functions as a checklist, ensuring that the writer knows what information to send.

\emph{Pros:} Could reduce the amount of missing information in the initial email to accounting or IT. 
Also leads to a better defined process.

\emph{Cons:} Requires additional work by recruiter. 
If template is too information heavy, some may ignore it.
Valcon employees dislike forms and rigid processes, so they may ignore it.

\emph{Conclusion:} Creating a template for OMT recruitment seems a good idea.
Creating one for Valcon would be a bad idea, as it would take time from management and consultants, but creating one for Hanne might work, as she would be reminded of the information required.

\subsection{Change of attitude toward IT}
\emph{Description:} Change the company attitude toward IT, from the current "they are there to fix my problems" to "they are there if I really need them".
We do not know how to implement this, but as Valcon is a consulting company, we assume they have a strategy for this case.

\emph{Pros:} May reduce the number of edge cases, recruiters would be more likely to send recruitment information in early, or give warnings that a quick recruitment is coming up.
Doesn't require extra time from recruiters.

\emph{Cons:} 

\subsection{Buffer of computers}
\emph{Description:}
\emph{Pros:}
\emph{Cons:}
\emph{Cost/Benefit:}
\emph{Conclusion:}

\subsection{New HR system}
\emph{Description:}
\emph{Pros:}
\emph{Cons:}
\emph{Cost/Benefit:}
\emph{Conclusion:}
