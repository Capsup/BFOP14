In this section we will explain our overall vision, describe and analyze our solutions, recommend an implementation strategy and finally make some recommendations for future analysis.

\section{Vision for overall change}
IT and Accounting are frustrated by the recruitment process and feel that time is wasted on it.
Our solutions aim to alleviate these frustrations and reduce the time spent on the process.

Instead of a single solution to the problem, we have decided to look at many small solutions, and recommend implementing many of them.
This is because the Valcon Group is two companies, and solutions that fit OMT do not necessarily fit Valcon Consulting.
We don't want to throw the current process away, but instead want to improve it in small ways to be able to scale with the company growth and possibly save some money.
Finally, we weigh work enjoyment and appreciation highly, as these are important for employee joy, integrity and competence, three of the Valcon Group's key values.

\section{Solutions}
Please refer to appendix \ref{app:solution_propositions} (p. \pageref{app:solution_propositions}) for a complete list of solution propositions.
Each solution contains a description, some positive and negative effects, a financial estimate (refer to appendix \ref{app:cost_benefit_analysis} for the cost/benefit analysis) and a conclusion describing it's suitability.

\subsection{Template for recruitment}
\emph{Description:} Create a template for recruitment so recruiters know what information is required. A template also functions as a checklist, ensuring that the writer knows what information to send.

\emph{Pros:} Could reduce the amount of missing information in the initial email to accounting or IT. 
Also leads to a better defined process.

\emph{Cons:} Requires additional work by recruiter. 
If template is too information heavy, some may ignore it.
Valcon employees dislike forms and rigid processes, so they may ignore it.

\subsubsection{Finance}
This solution would save appr. 6.500 DKK in the first year and appr. 8.000 anually afterwards, including initial costs and maintenance.

\subsubsection{Risks}
Should the template get out of date, irrelevant information would have to be sorted through everytime making it take longer time and defeat the purpose of the template. 
It's also a risk that the template won't be used, since it's work waste on implementing the template then and the process remains ineffective.

\subsubsection{Handling the risks}
To deal with the risks, it's important to revise the template every once in a while, every 3 months could be an idea.
To keep the recruiters interested in using the template, it might be relevant to remind them that the template makes the recruiting process smoother and faster, as long as the template is used properly.

\subsubsection{Conclusion} Creating a template for OMT recruitment seems a good idea.
Creating one for Valcon would be a bad idea, as it would take time from management and consultants, but creating one for Hanne might work, as she would be reminded of the information required.

\subsection{Change of attitude toward IT}
\emph{Description:} Change the company attitude toward IT, from the current "they are there to fix my problems" to "they are there if I really need them".
Could be implemented with a small, internal campaign to improve relations to IT, and to make IT feel more appreciated.

\emph{Pros:} May reduce the number of edge cases, as recruiters would be more likely to send recruitment information in early or give warning that a quick recruitment is coming up.
Doesn't require extra time from recruiters.

\emph{Cons:} It would be hard to measure the success of the change - when is an attitude changed?
There might not be a financial gain.

\subsubsection{Finance} 
This solution would probably not save money, but the non-financial benefits are important and the cost is low (appr. 20.000 DKK once.)
Some money may be saved by reducing the amount of edge cases.

\subsubsection{Risks}
It might not be possible to actually change the attitude that employees have towards IT already. Not only that, but putting focus on that incorrectly, could make the attitude towards IT even worse.

\subsubsection{Handling risks}
Should the risks mentioned above be realised for this solution, a way to handle it would be to launch another campaign, effectively trying again. 

\subsubsection{Conclusion} 
We would recommend implementing this change. 
The reduced level of frustration in IT and greater work enjoyment would be worth it.

\subsection{Buffer of computers}
\emph{Description:} Keep 2-3 pre-installed computers ready for use at OMT and at Valcon, for emergencies or very quick recruitments.

\emph{Pros:} Would reduce response times to urgent recruitments.
Also useful if an employee's computer breaks, to reduce downtime.

\emph{Cons:} Needs a little more managing from IT, as
computers will need to be kept updated and ready.
As all computers are not in use, this creates a small overhead.

\subsubsection{Finance} 
This solution has a big initial investment (240.000 DKK) and will not break even unless errors happen.
The money saved comes from employees being able to get back to work quickly after a computer breaks down, or letting new employees start work faster.

\subsubsection{Risks}
Maintaining a buffer of computers could result in the computers getting outdated in case the buffer is too big compared to the amount of new hires.

\subsubsection{Handling risks}
Reducing the amount of computers kept in the buffer will reduce the risk of them getting outdated

\subsubsection{Conclusion}
Whether this solution should be implemented depends on how many errors happen.
We have not been able to get an estimate of how much time is lost due to computer errors or new employees waiting for computers.
As it would increase work enjoyment in IT and help with errors, we would recommend implementing this even with a small financial loss.

\subsection{New HR system}
\emph{Description:} Acquire a new HR system to facilitate communication between accounting, IT, and recruiters.
Valcon is already looking at HR systems.

\emph{Pros:} Recruitment process becomes more standardized, as it always follows the same path.
Information becomes more consistent, as personal spreadsheets are less needed. 
This also safeguards against employees leaving.
Errors are less likely, as number of manual entries of information is reduced.

\emph{Cons:} Costly to implement, requires training of accounting, IT, and recruiter employees.
Also requires maintenance.

\subsubsection{Finance}
We have been unable to make an estimate of the price of a new HR system.
\todo{Maybe we can make this estimate?}

\subsubsection{Risks}
If the new HR system isn't compatible with current systems and processes, it could convolute the process even further. 
If the new HR system isn't optimal and remains unused even after being introduced, it would just be yet another system having to be maintained and introduces additional completely in the process.

\subsubsection{Handling risks}
A major part in handling the risks associated with this solution is analysing the new HR system and making it absolutely sure that it's compatible with the current systems and the entire process. 
Introducing the new system incrementally could also reduce the risks associated as it would be possible to receive feedback early and change the system accordingly. 
Training the employees in the new system is also necessary to reduce the risks involved, as employees who doesn't know the system entirely, may not be able to use it properly.

\subsubsection{Conclusion} 
This solution is very necessary, as it is the only way information becomes more consistent.

\subsection{Other small improvements}
The following is a list of minor changes that would improve the process.
They have very little financial impact.
\begin{itemize}
\item Put a phone and internet questionnaire in contract letter, asking for the information the IT department needs.

\emph{Pros and cons:} Would reduce the need for contact between the IT department and the new employee.
Would require a little maintenance.

\item Create two "Unknown Department" departments in AD, one for Valcon and one for OMT with relevant access rights. The IT department can then put employees whose department is unknown into these, and have a common place to look and update.

\emph{Pros and cons:} Would create a work flow for non-standard process and thereby reduce errors. Would require some maintenance.

\item Have OMT recruiters send a mail when they know in advance someone will be needed, but before enough information is available.

\emph{Pros and cons:} IT can use the information to plan ahead, setting a computer and an employee aside when they believe the information will be there. Will reduce frustration in the IT department. Requires a little time and some planning from OMT recruiters.
\end{itemize}

\section{Implementation strategy}
To implement the ideas explained in this section, we suggest that Valcon takes a 'plunge' approach to the implementation, meaning the entire vision of the overall change is implemented all at once.
We see no reason to incrementally introduce the solutions to the company, since none of them are mutually exclusive and can easily be introduced at the same time since none of the solutions carry much risk and it reduces the amount of time necessary for the employees to get accustomed to the new processes.

\section{Future recommendations}
The following is two recommendations for future projects that may discover more things to improve in the recruitment process.

\subsection{Analyze OMT process}
Conduct an analysis of whether the OMT recruitment process can be improved.
Investigate whether they can know further in advance who and when someone is needed, so the number of urgent cases can be reduced.

\subsection{Further automation of install process}
The install process is very time consuming, and is a candidate for automation.
Looking into whether more of it could be automated could save time.
