Change the company attitude toward IT, from the current "they are there to fix all my problems" to "they are there if I really need them".
Could be implemented with a small, internal campaign to improve relations to IT, and to make IT feel more appreciated.

\subsubsection{Pros} May reduce the number of edge cases, as recruiters would be more likely to send recruitment information in early or give warning that a quick recruitment is coming up.
Doesn't require extra time from recruiters.

\subsubsection{Cons} It would be hard to measure the success of the change - when is an attitude changed?
There might not be a financial gain.

\subsubsection{Finance} 
This solution would probably not save money, but the non-financial benefits are important and the cost is low (appr. 20.000 DKK once.)
Some money may be saved by reducing the amount of edge cases.

\subsubsection{Risks}
It might not be possible to actually change the attitude that employees have towards IT already. Not only that, but putting focus on that incorrectly, could make the attitude towards IT even worse.

\subsubsection{Handling risks}
Should the risks mentioned above be realised for this solution, a way to handle it would be to launch another campaign, effectively trying again. 

\subsubsection{Conclusion} 
We would recommend implementing this change. 
The reduced level of frustration in IT and greater work enjoyment would be worth it.