\emph{Description:} Acquire a new HR system to facilitate communication between accounting, IT, and recruiters.
Valcon is already looking at HR systems.

\emph{Pros:} Recruitment process is able to become more standardized, as it will potentially be able to always follows the same path.
Information becomes more consistent, as personal spreadsheets are less needed. 
This also safeguards against key employees leaving.
Errors are less likely, as number of manual entries of information is reduced.
It opens up the possibility of automating several key processes, reducing workload and wasted time further.

\emph{Cons:} Costly to implement, requires training of accounting, IT, and recruiter employees.
Also requires maintenance.

\subsubsection{Finance}
We have been unable to make an estimate of the price of a new HR system.
\todo{Maybe we can make this estimate?}

\subsubsection{Risks}
If the new HR system isn't compatible with current systems and processes, or can't completely replace old systems such as QHR, it could convolute the process even further. 
If the new HR system isn't optimal and remains unused even after being introduced, it would just be yet another system having to be maintained and introduces additional complexity in the process.

\subsubsection{Handling risks}
A major part in handling the risks associated with this solution is analysing the requirements for a new HR system and making absolutely sure that it's compatible with the current systems and the entire process. 
Introducing the new system incrementally could also reduce the risks associated as it would be possible to receive feedback early and change the system accordingly. 
Training the employees in using the new system is also necessary to reduce the risks involved, as employees who don't know the system entirely, may not be able to use it properly.

\subsubsection{Conclusion} 
This solution is very necessary, as it is the only way information becomes more consistent.
