\label{app:time_spent_on_the_problem}
These calculations are based on the recruitment and resignation data sheet in appendix \ref{app:recruitment_data}.

\section{Standard process and edge cases}
\begin{itemize}
\item In 2012/13, 118 persons were hired or subcontracted.
\item In 2013/14, 154 persons were hired or subcontracted.
\item In 2014/15, 80 persons have been hired or subcontracted so far.
\end{itemize}
These numbers show how many computers have been made ready by year.
\emph{There is a small increase in the number of computers being readied (30\% in 2013 and (projected) 4\% in 2014) in correlation with the amount of hires.}

\subsubsection{Time spent on computers and data entering}
Peter gave an estimate for how long it took him to set up a computer and enter data into the system, 45 minutes (appendix \quoteref{app:peter}{peter_estimat_setup}). (We realize Peter may be wrong about this. We haven't been able to observe the process, but we still think the calculations are useful.)

Similarly, Lisbeth estimates that she spends about 15 minutes entering data into the system. (appendix \quoteref{app:lisbeth}{lisbeth_time_spent})

This means, that 118 hours have been used in 2012/13, 154 hours in 2013/14, and 80 hours in 2014/15 so far on computer setup and data entering.

With about 150 hours used every year, even if we save 30 minutes for every recruitment, this is only 75 hours saved every year, so \emph{our proposed solutions need to be inexpensive.}

\subsubsection{Time spent on other things related to recruitment}
Peter uses time on gathering information on telephone and internet connection preferences from the new employees, appr. 10 minutes on average.
(appendix \quoteref{app:peter}{peter_estimat_contact})
This is only on the Danish recruitments, and not on subcontractors.

Furthermore, he uses time on Danish resignations, both subcontractors and hires, appr. 15 minutes on average.
(We realize Peter may be wrong about these. We haven't been able to observe the process, but we still think the calculations are useful.)

\subsubsection{Total time spent}
Here follows the total amount of time spent on recruitments and resignations each year (barring edge cases).
\begin{itemize}
\item 2012/13 - 140,5 hours.
\item 2013/14 - 186,5 hours.
\item 2014/15 - 95 hours so far. (Half the year has passed, so a reasonable estimate is that 190 hours will be used in total.)
\end{itemize}

As long as there are few edge cases, \emph{the time spent is a small problem.}

\subsubsection{Edge cases}
We haven't been able to get any numbers on the amount of edge cases in the process.

This is problematic, as our main focus is documenting that there is a problem, and the amount of time spent in normal cases is small.

\subsubsection{Benefits to improving the process}
There are some benefits to changing the recruitment setup process.

Currently, IT and accounting are both frustrated by the process.
By improving the process, the work enjoyment could improve.

Finally, improving the process can speed up response times on support and recruitment from IT and on recruitment in accounting.

\subsubsection{Conclusion of process analysis}
There is little financial gain in changing the recruitment setup process.

The benefits, however, are important and in line with the Valcon's employee values.

So the solutions we propose must be without or with very little cost.