\label{app:peter}
\begin{linenumbers*}
\subsection{Summary}
During the interview with Peter we got a thorough explanation of the process of hiring in the IT department. 
There's a general unhappiness with Lotus Notes and that IT at times become stressed, since they're asked to have a PC ready by tomorrow. 
He's not happy with the fact that IT have to make the initials for the new hires, since the HR department might as well do this on their own or atleast have it automised along with the rest of the process. 
The process should in general just become automised or atleast looked at, such that the process is under control and the company can continue with their plans of keeping up a higher growth rate.
It's mainly a problem for OMT hires as they're generally late at announcing the arrival of new hires and this puts IT into a position where they're forced to halt their other work to get the work done for the new hire.

\subsection{Interview notes}
Merrild introducerer os for Peter og fortæller lidt om hvad det er vi laver her ude

Peter siger at det ikke er en hel smooth proces

Merrild: Hvad sker der fra din synsvinkel når der bliver ansat en ny medarbejder?
Peter: De går selv ind og kigger i notes databasen for at se om der er blevet oprettet. Det skal gøres manuelt og det gør han i løbet af dagen på et tidspunkt.
Når det er fundet, går de igang med at oprette i AD og exchange og giver dem initialer og accounting responderer at initialer er i orden og de gør deres ting i maconomy etc.
\linelabel{peter_stamdata}
Derfra finder de en PC og kontakter dem for at spørger om stamdata er i orden og hvad for nogen andre services de vil have, inkl. ADSL og nyt telefon nr. 
I notes står deres navn, mail addresse og telefon nr. Han ved faktisk ikke hvor informationen kommer fra. Der står måske hvilken afdeling de skal ansættes i “hvis de er heldige”. \linelabel{peter_datamangel} Det er ikke altid at OMT ved hvilken afdeling de skal ansættes i, det ved OMT måske ikke fra starten af. Det er vigtigt fordi de ved så hvilke andre der sidder i afdelingen så de kan oprette dem med samme rettigheder.
Det er vigtigt at det oprettes rigtigt fra starten for at slippe for dobbelt arbejde. 

Han tager beslutningen om hvilken PC de skal vælge, det gør medarbejderne ikke mere. Det var for dyrt, OMT får altid 15 tommer samt designere. De kan få 12 eller 14 tommer hvis de skal ud og rejse meget. 
De kan få en android af firmaet eller en iPhone hvis de selv står for det. 

Merrild: Hvor lang tid tager hele den her proces for dig?
Nogen timer siger han. Det tager ikke mere end 5 minutter at oprette dem og 5 minutter til at checke om der er kommet nye. 10 minutter med ventetid og få oprettet i AD og exchange. Computer opsætning tager et par timer, \linelabel{peter_estimat_contact} tager 15-20 minutter at skrive til brugeren da det skal læses ekstra i gennem og ellers kopieres det. \linelabel{peter_estimat_setup} Udstyr samles i tasker, 30-60min + registrering. 1 time til intro dag hvor de får at vide hvordan de logger ind og andet. Cirka 3,5-4timer per ansættelse. 

Merrild: hvor tit?
I sidste uge kom der 4 medarbejdere og de sidder med 9 nu. Det tager en masse tid. Det er 5 fra Valcon og 4 fra OMT. 

\linelabel{peter_short_notice}
Valconerne er cirka 1 måned i god tid, hvorimod OMT f.eks. kom i går (d 20/10) [han nævnte både ‘i går’ og ‘i fredags’] og skal ansættes den 24/10. Men der er også hvor de fik 3 ugers advarsel. De kan godt komme en fredag og så forvente at have det klar mandag. \linelabel{peter_frustration1} Han mener ikke at de er særlig struktureret og de burde få at vide at det tager noget tid og få nogen regler for hvordan det skal foregå.

Merrild: I den relative korte tid du har være ansat, forestiller du dig så fremover at der kommer flere ansættelser?
Han har svært ved at se det stoppe fordi firmaet er i høj vækst. Han tror ikke at man kan blive ved med at få så meget vækst, men det er jo lykkedes meget godt indtil videre. Han mener primært at det er OMT som har rigtig meget vækst. Pludseligt har OMT ingen kontrakter og så ryger der en masse mennesker måske da de primært arbejder på projekt basis fordi de har en masser eksterne konsulenter. Så kommer der pludseligt et nyt projekt og så ansætter de en masse igen. Der ryger 100 og pludseligt ansættes 100 igen. Valcon, mener han, fortsætter stille og roligt med at ansætte nye.

Merrild: Vi har fået at vide, at nogen gange, i stedet for databasen så får I en email?
Det sker, siger han. De prøver at gøre hvad de kan. De sætter en PC over med det samme og checker QHR for et navn, så ellers så må de spørge ansætteren for et navn så de kan fortsætte procuderen som normalt. Email er normalt dem hvor det skal gå stærkt nævner han. De kan ikke nå at sende mail ud til folk fordi det skal være klart NU, der bliver lavet en PC og sendt afsted og så snakker de med ham når han bliver ansat.

De skal bare bruge et navn, da de så kan oprette på AD, Exchange og Afdeling (som de ikke altid får) så de kan lave initial etc.

Merrild: Sådan noget med ADSL og telefon, ville det være en fordel hvis I fik det med det samme?
\linelabel{peter_blanket}
Ja, det ville faktisk være en fordel. Det ville være helt genialt hvis de fik et helt ark hvor det bare skulle udfyldes så de allerede havde taget stilling til det. Han mener det burde stå i systemet allerede. Han mener det ville være helt perfekt, så skulle de bruge meget mindre tid på det og så når personen kommer, så er alting bare klar til dem. 

De sender blanketter til den email de får for portering af telefon, så derfor skal de vide om nummeret vil beholdes. De sender bare en email til dem med skemaerne så de selv skal finde ud af det. Det kan være et problem fordi de måske ikke får sendt den rigtige skema. Så skal de sende den igen så de kan gøre det igen. Det ville måske være optimalt at Valcon kendte til deres udbyder sådan at de kunne sende den rigtige blanket til dem med det samme, så de ikke selv skulle finde ud af hvad for en der skulle bruges.

Merrild: Hvad kan gøres bedre?
Minimum 3 dage eller 5 dage burde være en regel så IT kan nå det. Muligheden skal selvfølgelig også være der for meget vigtige ansættelser. \linelabel{peter_frustration2} De burde vide at når de ansætter en, så tager det så og så lang tid og det tid tager det bare. Han var tidligere i en virksomhed hvor hvis de sagde at hvis de kom i morgen, så havde de bare ikke noget udstyr. \linelabel{peter_frustration3} Selvom de sagtens kunne nå det, så skulle lederne lære at de ikke bare kan misbruge en anden afdeling. Så derfor lærte de dem på den hårde måde. Han mener det er en fin måde at lære huset på at sådan er processen bare. Det kunne være fedt hvis der kom et rigtigt HR system istedet for notes og QHR. Han udfylder ikke database og det andet, det gør studentermedarbejderne. Han mener slet ikke at det burde være nødvendigt, det burde bare ske automatisk når initialerne er oprettet. 
Idag checker han initialer, hvis de ikke er taget så sender han dem videre. 

Merrild: når du modtager initaler og opretter dem i AD’et, så søger du databasen først?
Ja, for at checke om der ligger noget gammelt skrald. AD’et og databasen ligger ikke sammen, så der kan være forskel på data’en i de to forskellige. Så derfor søger han i begge for at checke om initialerne er i orden. Der kan godt ligge årsvis gamle mail addresser og det er et problem når der så skal oprettes nyt. Der er scripts der bliver kørt, men det bliver kun kørt eksternt en gang i mellem og han ved egentlig ikke hvad det script gør præcist, men det er meningen at det skulle rydde op.

Merrild: Du vælger initialer baseret på et navn?
Hvis de har 3 navne, forbogstav x3. Han har ikke selv været ude for at det kunne give en uheldig kombination, men på sit gamle arbejde havde han. \linelabel{peter_initialer} Der ville han så ændre det, f.eks. kommer der ikke stil at stå “dick@valcon.dk”. Det er han professionel til. Han sidder selv og prøver at ændre initialerne \linelabel{peter_old_employees}og der kan det også være et problem hvis han ryger ind i initialer som rammer navne der eksisterer i forvejen. Så prøver han sig frem til et navn som passer, måske tager han så mellemnavn, fornavn etc. 

Merrild: Hvis en bruger slettes i AD’et, han er ikke længere ansat, så kan der stadig ligge en gammel mail addresse? 
Hvis der ligger en postkasse i forvejen, ville han backe den op først manuelt og så give den nye ansatte den samme postkasse.

Merrild: Du er master of excel sheetet?
Der står noget om hvornår de starter, initialer, navne, skal de have ADSL, mobil, hvilken PC, frederikke har noget til portalen og michael/mathias har noget QHR data. \linelabel{peter_information} QHR’eren er ikke altid up to date, det bliver først gjort når der er tid til det og det er der måske først tid til om 1 eller 2 måneder. Det er primært stamdata, hvis det opdateres i QHR så opdateres det andre steder. 
Den bruges også til at checke hvor langt de er i processen. Hvis ikke de er blevet kontaktet, skal de kontaktes eller afventer svar. Det holdes der også styr på manuelt. Hvis han fik det fra starten af, så ville han bare bruge QHR / HR systemet, så ville excel arket ikke nødvendigvis være nødvendigt. Hvis nu de fik data’en fra starten af, så kunne der bare være et hak for at IT havde styr på alt deres.

Han er tydeligvis ikke en fan af Lotus Notes, han hader det virkeligt. Han mener det burde have været ude da firmaet var mindre. 

Et eksempel var i hans tidligere firma, kommune data, så havde de 4 forskellige systemer til det. Oprettelses processen var en email skabelon som de sad og udfyldte og det var ikke holdbart mener han, men sådan startede det. Så gik de videre til andre systemer som de blev større. Han mener at alle burde bruge det samme system på tværs af organisationen. Hvis nu der bare var et link på portalen som der bare kunne trykkes på og udfylde data’en ud. Udfyld navnet så der kan oprettes initialer hvor systemer bare tjekker om initialerne eksisterer i forvejen og så opret det. Han mener det er mere professionelt hvis HR bare fandt initialer, det burde IT ikke. HR har styr på om en medarbejder er fin for firmaet at vise frem og derfor burde de gøre det.

Merrild: Hvem er HR?
Dem der er HR er Bjarne, Hanne og Camilla. De er faktisk ikke HR afdeling, men de ‘agerer’ HR. Han ved faktisk ikke hvad de er til daglig og ved ikke hvad de laver, men han ved at de ikke er sekretære men tror de er konsulent af en eller anden art. De står ihvertfald for ansættelser men han tror ikke at de har de andre normale HR opgaver. Medarbejder pleje og alt det der, det har de ikke tror han. 

Merrild: [Hvad skal lisbeth bruge?]
Hun skal bruge initialerne og navne og stamdata, men det får hun jo ikke før IT har oprettet det. 
Det burde være sådan hvor nu er der en eller anden ansat og så sætter lisbeth sig ned og indfører dataen istedet for at der bare skrives en mail til IT og så skal IT få fat i det. 

Merrild: Danny’s HR system, hvis der var en proces hvor informationen kommer fra lisbeth direkte til jer, hvor i så bare skulle godkende initialerne?
Han mener stadig at det burde være HR som gjorde det, men hvis et system bare indsatte alt det direkte i systemet og de bare fik en email, så ville det være virkelig godt.

De har en fælles postkasse som de bruger hvor de bare påsætter deres egne farver så de får opgaven. De har ikke rigtigt noget task management system, det ville jo være “for smart”. 

Han syntes det ville være fint hvis de bare fik en email hvor en person bare er blevet oprettet og de får den information de skal bruge tilsendt, så ville det være fint fordi så kan de bare gå igang med PC’en.

Merrild: Har I et system for hvilke resourcer de forskellige medarbejder har?
Ja …. Det er *også* i notes. (Han kan tydeligvis ikke lide notes) Det burde også ligge sammen i et fælles system. Han tror det har noget med penge at gøre at det ikke er blevet ændret. 
Han bruger også tid på at opdatere den data så de ved hvilke medarbejdere der har hvilke resourcer. Det gøres når tasken bliver givet ud. Det er helt klart lorte arbejde, så derfor får studenter medhjælperne den opgave. 

Merrild: Har I en database over hvilke ting I har på lager?
Når de får nye PC’ere, så sidder studenter medhjælperne og scanner dem ind og indsætter data’en i Lotus Notes. Når der så tages informationer, så sætter de bare det udstår som ‘udlånt’ til den person. Når de får det tilbage, så skal det selvfølgelig også registreres og opdateres. 

Merrild: Ville det spare tid for jer hvis I fik en liste over hvilke resourcer der skulle udlånes?
Ja, det ville det. Hvis den data var tilgængelig, så gør det det også lettere at få dem tilbage igen. Hvis informationen over hvad der var ledigt var tilgængelig så de bare fik at vide hvad der skulle gives ud, så ville det også være godt. Så de ikke behøver tage beslutninger, men at systemet bare siger at det og det skal gives ud, ville være godt.

Merrild: Har I nok system i lageret til at det ville være tidsbesparende?
Nej, så meget styr er der slet ikke på lageret til hvis systemet bare sagde: “Tag telefon nr 101”. Det er der ikke styr på i lageret til, så det ville nok ikke være tidsbesparende. 

Merrild: Alle de her små ting til registreringer
QHR: Information om medarbejder
Techadm: Systemet til de resourcer de har
Excel ark: Telefon nr. (ikke særlig snedigt)
AD: Information om medarbejder

Man burde have ét til alle systemer, det ville være optimalt. 

Merrild: Du snakkede noget om det her med OMT, bare smid det over på IT?
Det er nok ikke sådan de ser på det, men det er den generelle holdning at ‘det klare IT bare’. Men nu er firmaet blevet så stort og IT er ikke blevet større. Han kom ind som ekstra mand men så stoppede en og han stoppede pludseligt bare holdet. Afdelingerne agerer som de stadig kun er 15 mand i firmaet og det er de jo ikke længere. Så IT er fortsat i undertal.

En mindre historie om hvordan han ikke tror Maersk IT har det på den måde, men her var der regler for alting og når folkene fra OMT så kom ud derfra så var de glade fordi nu kom der over i et mindre firma hvor de bare kunne gøre det på den lette måde. \linelabel{peter_standardiseret_proces} Men han siger at det kan det ikke fortsætte med, fordi Valcon er også oppe i den størrelse hvor det nu er nødvendigt at have regler for de forskellige procedurer så man ved hvor lang tid de forskellige ting tager og at man er indforstået med det. 

De prioriterer Helpdesk sager, hvis der er et ITR (?) nummer. De har en frontline support som har deres eget sagtsystem, men det bruger Valcon jo ikke. Så Valcon siger bar at de har klaret en hvis sag og så styrer de deres eget sagsstyrings system. 

Der skal styr på systemet inden de bliver meget større, fordi der er det bare for sent. Så er der pludseligt 200 emails over en weekend og så er det umuligt at holde styr på det. 

Merrild: Vi kigger på det her med ansættelser, er det et generelt problem du føler det her med at afdelingerne bare tror at I kan klare alt?
Han føler primært at det er OMT som har et problem med det. Han tror ikke at han har styr på deres processer, de tror stadig at de bare kan lave en ansættelse og så er alting klart i morgen. De bliver nødt til at have nogen procedurer for hvor lang tid der kan gå før tingene er klar. Han mener at det måske er der hvor man burde starte. På dem virker det som om at de ikke har styr på tingene og at de så er ligeglade fordi de skal bare have medarbejdere NU. 
De ansætter dem i flæng og det gør det endnu sværere fordi pludselig er der 10 der skal ansættes.

Merrild: Hvor lang tid tager PC’ere?
De bestiller dem hos Dustin og hvis de er på lager, så har de dem inden for en dag. Hvis de ikke er, så tager det 3 dage. Hvis de er uheldige og de er på backlog, så tager det 7-14 dage. Generelt 3 dage hvis det tager lang tid. 

Merrild: Så hvis der kommer 7-9 inden for en uge, så bruger I 4 timer per person = 36 timer? Det kan nås på en uge?
Så er der ikke tid til andre opgaver, så det sker ikke i virkeligheden. 

Merrild: Er det din opfattelse at ansættelser tager prioritet?
Ja, det er de jo nød til. Fordi medarbejder SKAL have dem før de rigtigt kan bruges, de ansættes jo fordi de skal bruges. Hvis de bare får en blok papir og blyant i en uge, så er de jo ikke brugbare heller. Så venter de bare til torsdag og så kan Mathias tage sig af det, fordi der kommer han jo. 

Merrild: Er det sket at det bliver sagt at de skal starte i morgen?
De har også prøvet hvor de kommer ned og siger at de har en samtale nu, og at de gerne vil have en PC klar nu imens han sad der oppe. Så der lavede de en PC og han havde faktisk ikke engang skrevet under endnu.
De har også prøvet at få at vide at en person blev ansat for 13 dage siden og han undrer sig over hvad personen så har lavet i 13 dage når de ingen PC eller andet havde. Han mener ihvertfald OMT er mærkelig med deres ansættelser

Merrild: Er det Danny der siger i skal priotere ansættelser?
Nej, det er noget de bare ved. Sådan er forretningsgangen bare. Det skal jo selvfølgelig bare gøres. Det er en generel holdning i afdelingen tror han. 

De kører selv tingene til Odense, det gøre Allan normalt. Ellers så spørger de sekretariatet og spørger om der er nogen i Hørsholm som skal der over og så tager de tingene med. De bruger ikke penge på transport. 

Martin: Det her med stamdata, er det noget der skal bruges med det samme eller kan det vente for HR?
Det skal gerne indtastes inden at de starter så informationen kan bruges. Frederikke indtaster f.eks. data så det kan bruges i portalen. 
\end{linenumbers*}