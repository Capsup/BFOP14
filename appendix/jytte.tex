\begin{linenumbers*}
\subsection{Summary}
Jytte believes it would be beneficial to have a standard form for recruiters in OMT to fill out.
The standard form should contain fields for all the information required by accounting and IT.
Some information should not be on the form, e.g. salary information, because it should be shared with as few as possible.
Jytte doesn't spend much time communicating with Lisbeth about new employees.
She doesn't believe that the hiring process will happen more often, because it has never been a strategy for OMT to grow as much as they have.
On the other hand they will continue to let go of and rehire subcontractors.

Jytte also believes that it would be a good idea to have a supply of PCs and mobile phones in OMT, as well as having an IT employee at OMT every day, instead of the ~3 days eveyr week that is currently there, to improve response time.

\subsection{Transcript}
\label{app:jytte}
Michael:
Lidt om det vi laver. Vi har et stort fag, BFOP.

Jytte:
Hvad læser I?

Michael:
SWU[Software udvikling], men ITU vil gerne have nogle udviklere
der kan andet end kode.
[Forklarer om hvad BFOP og projektet går ud på]
Danni foreslog ansættelsen processen. 
[Forklarer de dele af processen som vi kender til]
Vores opgave er at lave buzzcase. Vi har snakket med Peter,
Lisbeth og Hanne. Vi vil gerne høre hvad processen er i OMT?
Vi har hørt det er dig.

Jytte:
Det er sandhed med gran salt. Alt hvad der er
Underleverandører, skæve, midt i måneden, ikke standard.
Hvis ikke man har en standard platform, så i hvert fald
en standard template. 
Hvordan bestiller man ny medarb. En template der dækker hovedparten.
\linelabel{jytte_standardised}
Vi deler ikke ansættelsesforhold, løn bonusser etc. med IT.
Nogle ting skal deles med alle. De ting kan lige så godt være
standardformularer.
Vi er ISO certifiseret. Der ligger mange standardformularer,
den kunne ligge samme sted.

Michael:
Som fastansættelser, hvordan er processen i OMT.

Jytte:
Lige som Valcon, hvis ikke personen kendes så godt så første interview. 
Synes man match så tests. Bruger ekstern HR konsulent til at klare tests.
Det der ikke kører per automatik. Ligge DISC profiler ind på CV. 
Har for nyligt formidlet til Hanne om hun ikke kunne ligge dem ud.

Michael:
Så info til Lisbeth kommer ud efter sidste samtale?

Jytte:
Efter tests så ny samtale. Hvis ansat så bestil kontrakt hos Lisbeth.
Hun får al data fra mig.
Så kommer den i papir fra Valcon. Nogle gange som PDF.
Det behøver ikke standardiseres. 
Så sender vi svar kuverter med. Laver en aftale med ansatte
om at de skal sende til Lisbeth.
Men når de har sendt så bedes de sende en mail til mig. 
Når Lisbeth modtager kuvert, så sender hun også mail til Jytte.
Det sker at kuvert ikke når frem til tiden.
Det skal være fast procedure at Lisbeth bekræfter at hun har modtaget kuvert.

Michael:
Hvor meget er du med når det er eksterne kandidater?

Jytte:
Det varierer. Nogle gange så ringer fo
[faglederen til mig of fortæller at der er en på vej. 
Andre gange så får jeg ikke noget af vide.]

Michael:
Hvem bestiller?

Jytte:
\linelabel{jytte_stamdata}
Faglederen. Vi har nogle gange pladsproblemer i OMT.
Jeg mangler også at få information omkring nye underlevenrandører fra faglederen. 
Der er skal være en standard formular som skal udfyldes med stamdata
+ hvilken PC der skal bruges. 
Jeg tror I bruger 3 kategorier(af pc)

Michael:
Jo, vi har 3 men det er også ved at blive lavet om afhængigt af lager.
Det her med at sende info til Lisbeth,
bestille kontrakt osv er det noget du bruger meget tid på?

Jytte:
Nej. Har efterhånden bestilt så mange. God kommunikation.

Michael:
Kommer denne her process til at ske oftere?

Jytte:
Nej, det tror jeg simpelthen ikke på. Men det har jeg jo sagt før.
Det har aldrig været vores strategi.
Så skal vi til at vokse her(Hørsholm), men det har vist sig svært.

Michael:
Oplever du nogle gange det ikke går stærkt nok?

Jytte:
Ja, men det er jo svært. Der er altid nogen der ikke synes det går stærkt nok.
Hvis vi skal forbedre noget, så skal vi have et standard lager med PC og telefon.
Problem at Allan ikke altid er i Odense.
Vi har jo også snakket med Rasmus og Danni om det, ved godt at der er lidt presset i IT.
Når der kommer en ansat mere i IT så kommer Allan mere til Odense.

Michael:
Vi har snakket om lager af færdige PC’ere. Så det kan gå hurtigere nogle gange.

Jytte:
Ja, nu er der jo det eksempel når der skulle gå til at ...

Michael:
Nyt HR system?

Jytte:
Ikke hørt noget, OMT ikke en del af det nye HR system.
Enten skal OMT bare have det, eller også skal de ikke være en del af det.

Michael:
Det der er snak om for at lette komm mellem Hanne og Lisbeth,
er et program der ville være. Smart?

Jytte:
Det løser ikke vores problem, for underlevenrandører ikke i det system.
Hvis man skal bruge sådan et system så bliver jeg flaskehals.
Jeg skal ikke være HR ikke mit job, vi har bare ikke nogen HR. 

Michael:
Hvad er dit job?

Jytte:
Ansvarlig for projektservices her, OMT og Kina.
Det er ikke mit job at sørge for at der er borde til folk.
Har ikke lyst til at være flaskehals for bestilling af PC’ere til nye underleverandører. 

Michael:
Der er ikke nogen ...

Jytte:
Pludselig en hel masse. Jeg kan sagtens sige at alle skal udfylde den her blanket,
men hvis sekretær skal indtaste det ikke godt. Jo færre der ved noget om løn jo bedre.

Michael:
Mest underleverandører der er mega travle?

Jytte:
Ja, der er undtagelser med fastansættelser, men det er ikke ofte.
For mange kokke om maden, 
Hvis vi skal bruge QHR(nye HR system) så skal vi ind over.
Det skal ikke være lige som portalen. Vi bruger ikke portalen.
 
Michael:
Bruger i filesync?

Jytte:
Meget meget lidt. Vi tjekker ikke meget ud, og må ikke.
Vi får ikke lært at bruge dem.
75\% af vores medarbejdere rejser ikke. De møder ind på kontoret hver dag og arbejder.
Hvis der er et HR system der kan bruges til gavn for Lisbeth og IT
så vil jeg gerne være med til en dialog om det. 
Men vi gider ikke det bølv der er med 

Michael:
Recap

Jytte:
Tving underlevenrandører til at svare det på alting.
Vi udlevere ikke tlf til underlev.
Nogle gange får de ikke tlf. med fra arbejdsgiver, så skal de have fra os(vikar).

Michael:
Mere IT i Odense en god ting?

Jytte:
Business objects rapport - det er derfor underleverandører skal i korrekt department

Merrild:
Hvor længe varer en ansættelse?

Jytte:
3 måneder, 2 år, det varierer. De må kun arbejde på vores platform, 
vores servere, vores PC, fordi det er et lukket projekt.
Det er derfor alle skal have en PC. Ingen clouds involveret.
Det er i høj grad på grund af det Canadiske projekt
at vi har brug for så mange underleverandører der skal have PC’er. 
På trods af det går vi meget efter lignende projekter.
Der er rigtig rigtig mange penge i det i forhold til kommercielle skibe.
Så det kommer til at ske igen.
Grunden til at OMT ikke vokser mere, er fordi der ikke findes flere i Danmark
med den viden om skibsbyggeri. 
Jyttes bud er at der er cirka 55 underleverandører nu.
Vi forventer at hvis der kommer et nyt projekt bliver der ringet til dem igen,
og så skal de have PC igen.
Måske lade være med at slette eksterne accounts,
lukke dem op igen når de kommer tilbage.

Det bliver et issue at hardware bliver forældet,
når alle sendes hjem når projektet er færdigt.

Konkurrenter:
Rosteen Knudsen, Knud E Hansen, HC consult, Shipcon. Principia North i Svendborg. 
Logamatik og Wilhelmsen kommer der el og automatik kompetence fra.
Vi bruger nogen fra : Process engineering. De laver rør.

KEH vil hellere lave roskibe og færger end de vil lave containerskibe,
og OMT vil ikke lave færger. Så det er ikke fordi der er konkurrence mellem
firmaerne. 
Konkurrence om de gode medarbejdere, men ikke andet end det.

Kategori vi ikke har snakket om: timelønnede.
De har ikke noget garanteret timetal, men de skal også have en PC.
De får normalt kun en PC. Er det en som bare kommer ind til f.eks.
et værfts-optimerings-projekt som varer 3 måneder,
så skal de også have PC i bare 3 måneder.

Underleverandører får ikke løn fra os. Vi får en faktura fra dem. 
Med timelønnede håndterer vi betaling af dem, dvs Lisbeth.

Eksterne får et X foran deres initialer. Jeg ville hedde XMM, f.eks.
Når der søges på X kommer alle eksterne frem, og en gut der hedder Alex.
Måske et issue.

De bruger business objects til at se hvor mange timer
de forskellige medarbejdere er belastet hver måned.
Så det er godt at kunne sortere efter X’er.
\end{linenumbers*}