In this chapter, we list all identified tasks from the Process Chart appendix above (appendix \ref*{app:ProcessChart}), and list whether the item is currently an automated process. 

\begin{itemize}
	\item{Lisbeth is contacted with information of new employee}\\
		Recruiters have to manually write emails to Lisbeth for every new hire. This process will never be fully automated, but there is room for improvement, as information to Lisbeth is currently sent completely manually, with no predefined process. \todo{De skal manually skrive mail til Lisbeth, men information skal sendes 'completely manually'? Bør vi ikke bare skrive at der er ingen defineret proces, så derfor er det mere eller mindre tilfældigt hvilke informationer der kommer med}
		
	\item{Contact is established to get missing particulars}\\
		Information gained both from OMT and Valcon is not standardised, meaning that Lisbeth has to manually identify missing particulars and ask for them. Asking for missing particulars will understandably never be fully automated, as there is a human element, but a better HR system could possibly automate part of the requisition by identifying missing information and automatically fill out part of an email template.
		
	\item{Create new employee in QHR}\\
		Lisbeth has to manually paste provided data from emails to the QHR system to create a new employee. This probably won't be changed as long as QHR is used. A new HR system could be designed to automatically fill in data from emails.
	
	\item{Contract is created and sent to new employee}\\
		Not currently automated, but has a standard process (contract templates are filled out, printed and sent).
		With a new HR system, a contract template could be filled automatically.
	
	\item{Signed contract from new employee is received}\\
		The contract has to be manually scanned and the physical copy filed. No automation possible.
	
	\item{Lisbeth sets up new employee in Maconomy and the relevant payment system, based on information in QHR}\\
		Not automated. Data from QHR is manually copied to both maconomy and the relevant payment system for the user.
		As long as Maconomy and payment systems remain inaccessible to us\todo{Hvem er 'vi'? Projekt gruppen?}, automation won't be possible, even with a new HR system to replace QHR.
	
	\item{IT is notified that new information is available}\\
		Lisbeth manually sends a mail to the IT department, or verbally tells a member of the IT staff that a new employee is on the way.
		With a new HR system, IT could possibly be notified automatically with a button press in the program.		
	
	\item{IT checks QHR for particulars}\\
		Not automated. Done both manually, just to check, and when Lisbeth informs IT of new hires.
		Definite potential for automation for the same reasons as above.
	
	\item{IT determines initials and updates them in QHR}\\
		IT employees decide initials and update QHR with them. Initials could be found automatically from the new employees name. QHR does not support updating automatically, but a new HR system might.
	
	\item{IT notifies Lisbeth regarding initials}\\
		Not automated. IT notifies Lisbeth manually. This could also be done in a new HR system.
	
	\item{IT contacts new employee regarding new telephone, internet and subscription}\\
		An email template is manually filled out by IT. The process could be improved marginally by having a new HR system fill in the email template automatically.
	
	\item{Data is put into TechAdm}\\
		Not automated. Equipment is manually found and registered in TechAdm.
		This can't be changed as long as TechAdm is used. A new HR system might help with an overview, but equipment still has to be manually registered.
	
	\item{IT contacts Telenor and sets up new employee}\\
		A template is filled out by IT and sent to Telenor. There isn't a huge opportunity for automation, but the template could be auto-filled from a new HR system.
	
	\item{IT sets up new employee in the AD and their mailbox in Exchange}\\
		Information from QHR is manually pasted into a new AD account. There is a big opportunity for automation here, if a new HR system is introduced. AD user creation could pretty much be automated, as all the data is already there, it just needs to be copied the right places.
	
	\item{IT sets up a PC for new employee and sets it up in TechAdm}\\
		Partially automated / might be space for improvement. A large part of Windows installation happens automatically, but there is still a manual process involved.
		Setting up PCs might be automated further, but this falls out of scope for this report.
	
\end{itemize}