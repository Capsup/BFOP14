\label{app:hanne_interview}
\begin{linenumbers*}
\subsection{Summary of Interview}
Hanne is partly responsible for all hires in Valcon. She manages job interviews, DISC profiles and cognitive tests (partially), as well as planning meetings and communication between Valcon and potential new hires. \newline
Currently, this process involves a lot of manual data entry in a private excel sheet. 
Most of Hannes process is outside of the scope of this assignment, however a few key points from the interview were important:
\begin{itemize}
	\item{The hiring process in Valcon is sometimes chaotic, with consultants that start on short notice, or with key information missing. It should be possible to make PC's for new hires anyway, as their time is very precious}
	\item{Hanne manually updates a lot of information in a private excel sheet. A lot of this information is shared with Lisbeth, and could be moved to a shared system. This would also reduce Valcons dependency on a few key persons.}
	\item{Sometimes, the reason Hanne does not have the needed information for Lisbeth is because the general attitude amongst some consultants is that it isn't urgent}
	\item Hanne is not an actual recruiter at Valcon, however she is a big part of the process.
\end{itemize}
Furthermore, it was mentioned that no further tasks should be given to the consultants.

\subsection{Interview Notes}
Hannes process:
Som det er nu, er det rigtig håndholdt. Meget personafhængig. 

Der er forskellige veje ind. En ansøger kan kende en fra Valcon, sende CV til ham/hende, det ryger videre til job@valcon.dk som styres af Hanne.
Hun behandler alt - "tak for din ansøgning, vi kigger og vender tilbage til dig". Det er Hanne. Intet automatisk.
Hanne screener ansøgninger. Hvis de er tydeligt irrelevante, beslutter Hanne at sige nej tak direkte. Tvivlstilfælde sendes videre til Bjarke, som er recruitment manager, (han finder også folk fra linkedin og andre steder).
Hvis det måske kunne være interessant i en given afdeling, sendes CV'et videre til en chef for at høre hans mening. 

Hvis svaret er ja, så skrives der tilbage til ansøgeren "vi har kigget på dine papirer, vi vil gerne se dig til vores rekrutteringsproces til trin 1: et interview" Der møder man Bjarke, nogen gange også nogen fra forretningsområdet.
Her kigges der på personlighed. Det varer en times tid.
Hvis det er nej tak, så ringes der "nej tak". Hvis personen har været på Valcon, bliver det altid personlig henvendelse.

Hvis kandidaten er velegnet: kognitiv test og DISC test. Send link til DISC profil, invitér ansøgeren til at komme til test + tilbagemelding.
En uge fra link til møde. Det er altid Bjarke der tager mødet. Hanne laver den kognitive test. Bjarke er certificeret i DISC og kognitive test.

Der kan snakkes med Camilla Huus, hvis det er en spændende person men vi ikke lige ved hvor personen vil passe. 

Hvis det bliver go, så skal vi have et møde med den ansvarlige i det forretningsområde som det skal være i. En times tid, holdes af Hanne.
Hvis det går godt, sidste del. En casepræsentation, hvor Hanne sender en case til ansøgeren, og ansøgeren får en uge til at forberede casen.

Herefter et møde, hvor Camilla eller COO altid deltager, plus den ansvarlige for forretningsområdet, same Hanne eller Bjarke. Det er en form for rollespil.
Ansøger præsenterer i cirka 1,5 time. 
Herefter voterer de om hvordan det gik. 50\% går igennem, cirka. Casen er meget afgørende.

Efter det her er der et møde om løn, ansættelsesvilkår, salgsmål, bonusmål. Det skrives ned på et stykke papir som Lisbeth får. Hun laver kontrakt. Så er Hanne done.

Hvis afslag ringer en chef til ansøgeren og giver feedback på hvorfor det ikke gik. Det går rigtig godt. Hanne får søde mails fra folk som får nej tak.


Processen er flydende og tilfældig, den med information til Lisbeth. 
Mange forskellige variationer:

I den ene grøft, en konsulent kommer ned til Danni og siger "nu kommer Thorkild jo så i morgen. Der skal være en PC klar". Danni siger "hvem er han?" "ham har vi lige talt med så han starter om et par dage".
Samtidigt ind til Lisbeth "Ham her skal have en kontrakt lige nu, her er hans info. Lav en kontrakt."

I den anden grøft, efter kommandovejen. Efter voteringen sendes der en mail til Lisbeth fra Hanne. Hun [Hanne] skriver at der skal laves en standardkontrakt til XX. I mellemtiden findes salgsmål og bonusmål. Tilføjes i endelig udgave af kontrakten.
Der er standardkontrakt og så den endelige. 
Standard: efter case præsentationen, når vi ved han skal ansættes [en template, med basis information]
Endelig: [Når vi har fastsat løn og salgsmål]

[I] den gode process lægger Hanne CV over i Notes [QHR], Lisbeth får at vide hvad hun skal gøre, her er salgsmålet osv. Send en PDf til ansøger, så han kan se den før den printes og sendes.

Hvis Hanne skulle give et bud på det perfekte system:
Noget af det her håndholdte, skal ske mere automatiseret. men der skal stadig være mennesker til at udløse handlingen, of course. 
Efter casepræsentation, en besked til Hanne om at "vi vil ansætte sådan og sådan", Hanne ved at processen er afsluttet og kan registrere i sit excel ark, hun kan sende det videre til Lisbeth.

Somme tider svæver det lidt hos Hanne hvis ikke hun ved om de er endt med at blive ansat eller ej.

Der lægges et stamblad med bankkonto nummer ved kontrakten. Alt muligt skal udfyldes. Det er ikke alle der udfylder det. Og ofte haster det lidt eller meget. Ofte ringer man. Så det er [Rest of the sentence was lost]




Spørgsmål:
\linelabel{hanne_omIT}
Hvor sure vil folk blive hvis IT siger nej til PC om 2 dage
Svar:
Sådan en som Thorkild [Et eksempel vi fik på en ansættelse der skulle gå alt for hurtigt], han kom ud af det blå. Intet CV, intet interview, han skulle bare ansættes på grund af netværket. Vi ved at han er den helt rigtige til opgaven. Bum, han skal på et projekt lige nu.
Det er her vi adskiller os fra de meget firkantede hvor ting ikke kan lade sig gøre. 
Vi gør det fordi det er vigtigt for virksomheden, det er nødvendigt. Når jobbet er der, skal det bemandes.
Kunderne skal have konsulenter ud med det samme.

Hanne har jo også masser af jobsamtaler klokken 7 om aftenen. Vi må være fleksible.

\linelabel{hanne_buffer}
Brainstorm: IT har altid klargjort 3 PC'er som mangler at få "det sidste". Det belaster selvfølgelig vores cashflow, likviditet, IT.
Altså at lave en buffer så der er PC'er mere klar. 


Der er problem med dem som sender info for sent. Det er et problem alle steder. Hvordan får vi fat i dem.


Organisationen ønsker ikke at alle må tilgå systemet. Der må ikke lægges flere opgaver ud til konsulenterne. (det går imod Lisbeths drøm om indtastning af kurser osv.)
I den ideelle verden kunne en chef lave en rekvisition og bede om medarbejder, og alle var aktive i systemet. Men det kommer ikke til at ske, for han  [chefen] har for travlt.
Cheferne har for travlt. Det er en beslutning Rasmus har taget. "Det er godt at vide at det findes; det kommer ikke til at ske.". I hvert fald ikke mens Hanne er her.

Spørgsmål:
Hvorfor findes der ikke en standardiseret mail?
Svar:
\linelabel{hanne_standardised} Den vil ikke blive brugt. Af samme årsag som tidligere.  [at processen skal kunne være flydende]
Vi er i Valcon meget tilbageholdne med at bruge standarder.
Nogen ville være med på den, men der er flere som så sender en sms i stedet, for at komme uden om.

Organisationen er ikke klar til flere systemer. Det er nogen som vil være med på det, men der er for mange som ikke er.


Det som mindkey ville kunne:
Det kan afhjælpe et hav af de opdateringer Hanne laver på sit excel ark. 


Målet for Hanne:
I stedet for at skrive en mail til job@valcon.dk så henter man et skema og udfylder, som automatisk siger "tak for ansøgningen" med rigtig udfyldt data i mailen.
Hanne skal altid kunne give status på pipe, på disc, på case. Alle de her ting opdaterer hun selv. Hvis det kunne ske automatisk jo mere hjælp får jeg.
Men det må aldrig ske uden at Hanne har givet lov.
Bliver sommetider lidt overvældet over at skulle booke i 2 kalendre ved invitation til Disc, skal sende et link, bestille et nyt link, mange småting. 

Det kører godt for Bjarke - han har kun de funktioner han skal. men han blev også ansat til det; recruitment manager. Men Hanne skal rigtig rigtig mange ting.

\linelabel{hanne_losningomIT}
Det kunne også være en del af løsningen at kommunikere bedre om, hvad der er et stort pres på IT afdelingen. Så folk har lyst til at gøre ting i god tid.
\end{linenumbers*}