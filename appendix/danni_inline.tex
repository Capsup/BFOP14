\label{app:danni_inline}
\subsection{Summary}
During the interview, we show Danni the progress we've made and what we've come up with. 
He mentions that the size of Valcon is both a boon and a bane and explains Valcon's situation in the market and how they stay in the competition despite not being the biggest firm. 
Valcon are good at being with a project from initial to implementation and the projects they put out offers on don't see a lot of competition, so they generally know who they're competing with. 
OMT's choice of working with subcontractors is a conscious decision as it's part of a strategy for the company.

\subsection{Interview notes}
Michael: Hvad er vores forståelse af Valcon og hvordan vores forståelse af processen ser ud
Vi viser vores canvas frem og forklarer hvad det går ud på og hvad vi har skrevet
Danny foreslår at vi skriver ‘the how company’ på, det ville Poul være glad for
Michael: Vi har ikke fundet noget specifikt segment for Valcon
Danny: Der er også medico og rejseselskaber/flyselskaber. Det man kan sige er at public sector ihvertfald ikke er den største, det er industry/medico hvis man ser bort fra skibsdesign. Han syntes vi skal specificere at det er i 'no particular order' eller rykke public sector nederst ihvertfald.
Michael: [Det her er de] channels [vi har kigget på]
Danny: Relationship building er også en ting, dem der arbejder her ude har en kæmpe kontakt bog på 3500 kontakter som de kan hive på.
Michael: Key activities + key resourcer
Danny: I kan inkludere 'experience in how to' under key resources
Michael: Value propositions
Merrild: De der Kurser Valcon holder, er det substant?
Danny: Nej, det meste af det er udfakturering. Sales skal måske stå i bunden, da deres absolut største key activity er projekter.
Michael: Cost structure + key partners
Danny: Vi driver os selv, ja. Practical costs har en subkategori der hedder 'husleje'. Det ser ihvertfald godt ud.
Michael: SWOT analyse
Danny: SWOT analysen burde nok være på engelsk
Danny: Strength: Stor viden indenfor skibsdesign, lean og management
Michael: Valcon er sandsynligvis ikke de største? Er det korrekt?
Danny: Størrelsen af valcon er både en styrke og en svaghed. Vi har opgaver hvor vi kommer ud hvor Mckensy har lavet rapporter og analyser og så kommer vi og faktisk implementerer det, fordi det kan Mckensy ikke finde ud af.
Michael: Det at OMT skal bruge udefrakommende arbejdskraft er måske en weakness?
Danny: Det er helt klart en del af en strategi, fordi så kan de hurtigt udskifte arbejdskraften uden at skulle tænke på kontrakter og andet
Danny: Vi vil i virkeligheden gerne have flere store projekter, så hvis vi er hos DSB vil vi hellere lave flere medium størrelse projekter i stedet for ét stort, fordi de forskellige projekter kan yderligere udmunde sig i flere projekter
Michael: Der er ikke mange ligesom valcon
Danny: Lige præcist, 'the how to company'
Danny: Mckensy har rigtig mange unge ansatte, hvorimod valcon har mindre unge men rigtigt mange i midter segmentet. Så vi har mange der kan finde ud af at implementere tingene fordi de allerede har været ude i feltet, hvorimod Mckensy har mange unge som kommer direkte fra uddannelser og derfor er gode til at skrive rapporter
Danny: Der er mange ude at byde, men der er ikke mange om de bud som Valcon byder på. Det betyder at det generelt altid er de samme som byder og derfor kender de hinandens priser, svagheder osv.
Danny: Mckensy er større og derfor har de mulighed for at være competitive på priser og deres yngre konsulenter er billigere end de meget kvalificerede konsulenter som Valcon har.
Danny: Vi er selvfølgelig interesseret i at holde i vores konsulenter. Det gennemsnit som folk er i Valcon er generelt op til 5 år som ligesom grænsen for hvor det bliver svært at holde på konsulenterne. Mckensy har ikke noget imod bare at skifte deres klientel ud efter nogen år, de tager bare nye fra universiteterne
Michael: Forståelse af ansættelses processen, forklaring af vores flow chart, forklaring af vores interviews og hvad vi har fundet ud af
Danny: Kunne det være gavnligt hvis man med en graf at vise hvor mange reelle ansættelser det handler om?
Os: Ja, det ville være fedt med konkrete tal
Michael: Valcon kommer til at ansætte flere, det mener Jytte ikke at OMT kommer til. OMT kommer til at give slip på en større mængde mennesker, men ansætter dem senere igen
Danny: Hvad er next step for jer?
Merrild: Vi skal have skrevet alt det her ned og så skal vi have fundet faktiske løsninger. Derfor ville det hjælpe meget med konkrete tal
Danny: Det kan vi godt finde ud af, I kan bare spørge Lisbeth om det. Grafer og reelle tal mangler I generelt, det ville gøre det rigtig godt hvis I kunne få noget direkte substans i det med konkrete tal.
Danny: Cirka halvdelen af pengene kommer fra OMT
Merrild: Udfakturerer I forskelligt baseret på hvilke konsulenter der bliver brugt?
Danny: Ja, det gør de. Der er forskel på hvilke konsulenter der bliver solgt af tid for, så det betyder at der er dyrere konsulenter og mindre dyre. En konsulent fakturer imellem 1500 og 2800 i timen. Det dækker så også over flere ting, praktiske expenses inkl pc, transport, telefon og så videre.

