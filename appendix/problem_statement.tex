\label{app:problem_statement}
How can the Valcon Group save time and reduce frustration in the IT and Accounting departments by improving the process of new employee registration?
\\\\
Specifically:
\begin{itemize}
\item How can they reduce redundant work?
\item How can they automize routine tasks?
\item How can they reduce the amount of cases not following the standardized process, or refit the standardized process to accomodate the open structure of the company?
\end{itemize}

The following requirements need to be fulfilled for the project to be considered a success:
\begin{itemize}
\item
We must document the process to prove that there is a problem.
\begin{itemize}
\item If we prove that there is a problem, this requirement is fulfilled.
\end{itemize} 
\item
Having a standardized process reduces frustration and saves time, therefore communication should go through accounting every time, or standard process should be changed to allow for multiple paths.
\begin{itemize}
\item
Measure the amount of tasks currently not following standard process (mean/month)
\item
After implementation, measure the amount of tasks not following standard process (mean/month)
\item
If the number of tasks not following standard process after implementation is reduced by 50\%, this requirement is fulfilled.
\end{itemize}
\item
Tasks that could be automated, should be.
\begin{itemize}
\item
Look at the process, find all the tasks that could be automated and the ones that already are, make a list stating whether a task is manual or automated, and whether it must be manual or could be automated.
\item
After implementation, look at the list and compare the number of “could be automated” tasks to current process.
\item
If 50\% of the tasks that were previously not automated, but could be, are now automated, this requirement is fulfilled.
\end{itemize}
\item
The current speed of the process must be maintained or improved.
\begin{itemize}
\item
Measure the current mean time from first mail to accounting or IT, to employee is registered and IT tasks are done before and after implementation.
\item
If this time is the same or (preferably) faster, this requirement is fulfilled.
\end{itemize}
\end{itemize}