\label{app:solution_propositions}
\section{List of solution propositions}
\begin{enumerate}
\item Software that does it all
\item Template for recruitment
\item Change of attitude toward IT
\item Buffer of ready computers
\item Increase IT staff size
\item New HR-system
\item Help-program for finding initials and reading templates
	\item Other small improvements
	\begin{enumerate}
	\item Phone and internet paper in contract letter
	\item "Unkown department" in IT-systems
	\item Mail from OMT when recruitment is assured, but data is not yet available
	\end{enumerate}
\end{enumerate}

\section{Explanation of solution propositions}
\subsection{Software that does it all}
\emph{Description:} This proposal is for a piece of software that can encompass all the requirements of every employee in the company. Lotus Notes had this status once, but many disliked it.

\emph{Origin:}
When interviewing the accounting department it was noted that such a system would help out with a lot of the prevalent tasks.
(Source: appendix \quoteref{app:lisbeth}{lisbeth_system})

\noindent \emph{Pros:} Only one place for information, easy to manage.

\noindent \emph{Cons:} Very expensive, very hard to accomodate everybody.

\emph {Verdict:}
Not suitable. The solution is in direct opposition to the business strategy of the company, as it would require a large amount of concrete templates, training and time on behalf of the consultants.

\subsection{Template for recruitment}
\emph{Description:} When a new recruitment process starts, instead of a mail being sent with the information, a template is used to ensure that no information is missing.

\emph{Origin:}
In various interviews it was mentioned how lack of information damaged the recruitment process a lot, by increasing the overall work load required.
(Source: appendix \quoteref{app:lisbeth}{lisbeth_vende_tilbage} and appendix \quoteref{app:peter}{peter_information})

\noindent \emph{Pros:} No information missing means no need to contact employee more than once, saving time.

\noindent \emph{Cons:} Will put more work on recruiter, and Hanne stated that employees at Valcon dislike templates, so they won't use them.

\emph{Verdict:}
Partially suitable. The solution is in direct opposition with the Valcon business strategy, and thus we would not recommend it. However, based on interviews with the OMT partition of the company, templates would be possible for recruitments on their end. Thus we recommend it for the OMT group.

\subsection{Change of attitude toward IT}
\emph{Description:} Change the general attitude toward IT. Currently, IT is generally regarded as a department that just solves whatever problem one might have. Changing this attitude to better reflect the level of stress in IT could have benefits.

\emph{Origin}
When interviewing the IT department there was a notion that the company did not realise the amount of effort put into the work of the IT deparment, nor did most understand how complicating it was if recruitment requests were admitted with little to no spare time.
(Source: appendix \quoteref{app:peter}{peter_frustration3})

The other interviews backed up this notion, as they seemed to not know the implications of the IT departments work processes.
(Source: appendix \quoteref{app:lisbeth}{hanne_omIT} and appendix \quoteref{app:hanne_interview}{hanne_losningomIT})

\noindent \emph{Pros:} If people are more aware of how the IT department functions, they are more likely to be considerate. We think it would reduce the number of urgent cases and make IT more happy. Doesn't increase workload of consultants or management.

\noindent \emph{Cons:} Hard to implement, hard to measure.

\emph{Verdict}
Suitable. In order to maintain a high satisfaction for all employees, it is important that everyone understands each others work loads, and how it is increased under certain conditions. It would be easily achieved through generic information channels in the company.

\subsection{Buffer of ready computers}
\emph{Description:} Have a few number of computers installed and ready for use at Valcon and OMT.

\emph{Origin:}
In various interviews it was mentioned how this solution could ease the overall process, as it can deal with the situation of a critical recruitment.
(Sources: \quoteref{app:hanne_interview}{hanne_buffer} and \quoteref{app:jytte}{jytte_buffer})

\noindent \emph{Pros:} Better response times to urgent recruitments. Useful if an employee's computer breaks at a critical time.

\noindent \emph{Cons:} Needs a little more managing from IT, and maybe a full time IT employee at OMT. Computers will need to be updated. Resources not in use.

\emph{Verdict:}
Suitable. Having a buffer of computers would increase overall budget, but would solve many issues related to the recruitment process.

\subsection{Increase IT staff size}
\emph{Description:} Increase the IT staff size to increase throughput.

\emph{Origin:}
Baseline solution

\noindent \emph{Pros:} Makes IT capable of doing more.

\noindent \emph{Cons:} Expensive, and management increases.

\emph{Verdict:}
Not suitable. Simply increasing the size of the IT staff to accomodate demand is not a proper solution, as it would not increase the amount of capable workers. This is due to increased management of the IT department. The issues faced in the process cannot be reduced by simply having more employees working on it.

\subsection{New HR system}
\emph{Description:} Implement a new HR system for communication between accounting and employers.

Valcon are already looking at new HR systems.

\emph{Origin:}
In the interview with accounting, it was mentioned how the current HR system was deprecated, and not very user-friendly.
A new HR system would help the process in a way, that information has to be typed in fewer times.

\noindent \emph{Pros:} Process becomes more standardized and information more consistent. Copying data fewer times reduces probability of errors.

\noindent \emph{Cons:} Costly and requires training.

\emph{Verdict:}
Suitable. A new HR system would affect the company positively in relation to their business strategy.

\subsection{Help-program for finding initials and reading templates}
\emph{Description:} 
Write a small program that can read the template proposed above and enter the information into the IT systems.
Also able to propose initials for employees, and check whether the exchange mailbox has been used before.
Would reduce workload for IT and introduce fewer errors, but would require some maintenance and some time initially.

\emph{Origin:}
When interviewing the IT department it was mentioned how a tedious task it was to make up new initials. Since the making of initials are based on certain rules it is possible to generate them automatically. Furthermore they have to input data in various databases, as well as updating possible deprecated data.

\noindent \emph{Pros:} 
Less time used to making initials. Reduction of tedious tasks that must be conducted often.
Less time used on inputting data, reduction of possible typing errors.

\noindent \emph{Cons:} 
Generated initials might form unsuitable words, and the process has to be overridden by the employee.
System might be insufficient and not be capable of handling deprecated data correctly. 

\emph{Verdict:}
Partially suitable. The initials systems can be implement fairly easy. It can be taken into account that it must be possible to manually input initials if the generated initials are not sufficient.
The automated system will be too difficult to manage.

\subsection{Other small improvements}
\begin{itemize}
	\item Phone and internet paper in contract letter\\
	
			Put a paper in the contract letter sent to new employees, asking for their internet and phone preferences. Give it to IT upon return. Would help reduce missing data for new employees, and save time in IT.
			(Source: \quoteref{app:peter}{peter_blanket})			
	
	\item "Unkown department" in IT-systems\\
	
			Create a department for unknown departments in the IT systems, possibly both for Valcon and OMT, and create a process for managing it.
			Would remove the department requirement from IT, allowing for quicker recruitment.
			
	\item Mail from OMT when recruitment is assured, but data is not yet available\\
	
			Have OMT write a mail to IT when they know they will hire an employee soon, but don't have the data yet, so IT can plan accordingly.
			Would reduce stress, but increase planning for OMT.
\end{itemize}