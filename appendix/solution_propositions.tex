\label{app:solution_propositions}
\section{List of proposed solutions}
\begin{enumerate}
\item Software that does it all
\item Template for recruitment
\item Adjust expectations between IT and Valcon
\item Buffer of ready computers
\item Increase IT staff size
\item New HR-system
\item Help-program for finding initials and reading templates
	\item Other small improvements
	\begin{enumerate}
	\item Phone and internet paper in contract letter
	\item "Unkown department" in IT-systems
	\item Mail from OMT when recruitment is assured, but data is not yet available
	\end{enumerate}
\end{enumerate}

\section{Explanation of solutions}
\subsection{Software that does it all}
\emph{Description:} This proposal is for a piece of software that can encompass all the requirements of every employee in the company. Lotus Notes had this status once, but many disliked it.

\emph{Origin:}
When interviewing the accounting department it was noted that such a system would help out with a lot of the prevalent tasks.
(Source: appendix \quoteref{app:lisbeth}{lisbeth_system})

\noindent \emph{Pros:} Only one place for information, easy to manage.

\noindent \emph{Cons:} Very expensive, very hard to accomodate everybody.

\emph {Verdict:}
Not suitable. The solution is in direct opposition to the business strategy of the company, as it would require a large amount of concrete templates, training and time on behalf of the consultants.

\subsection{Template for recruitment}
\emph{Description:} Create a template for recruitment, that also acts as a checklist, so recruiters know what information is required.

\emph{Origin:}
In various interviews it was mentioned how lack of information damaged the recruitment process a lot, by increasing the overall work load required.
(Source: appendix \quoteref{app:lisbeth}{lisbeth_vende_tilbage} and appendix \quoteref{app:peter}{peter_information})

\noindent \emph{Pros:} Could reduce the amount of missing information in the initial email to Accounting or IT. 
Also leads to a better defined process.

\noindent \emph{Cons:} Requires additional work by recruiter.
Must be maintained.

\emph{Verdict:}
Partially suitable. The solution is in direct opposition with the Valcon business strategy, and thus we would not recommend it. However, based on interviews with the OMT partition of the company, templates would be possible for recruitments on their end. Thus we recommend it for the OMT group.

\subsection{Adjust expectations between IT and Valcon}
\emph{Description:} Currently, IT is generally regarded as a department that just solves whatever problem one might have. Some employees in the IT department do not agree. Adjusting the expectations between the IT department and Valcon could have benefits.

\emph{Origin:}
When interviewing the IT department there was a notion that the company did not realise the amount of effort put into the work of the IT department, nor did most understand how complicated it was if recruitment requests were admitted with little to no spare time.
(Source: appendix \quoteref{app:peter}{peter_frustration3})

The other interviews backed up this notion, as they seemed to not know the implications of the IT department's work processes.
(Source: appendix \quoteref{app:lisbeth}{hanne_omIT} and appendix \quoteref{app:hanne_interview}{hanne_losningomIT})

\noindent \emph{Pros:}
If people are more aware of how the IT department functions, they are more likely to be considerate. If the IT department are more aware of how Valcon functions, they will better understand why their work is critical. This would improve job satisfaction.

\noindent \emph{Cons:} Difficult to implement and measure. 

\emph{Verdict:}
Suitable. In order to maintain a high satisfaction for all employees, it is important that everyone understands each others work loads, and how it is increased under certain conditions. It could be achieved through generic information channels in the company.

\subsection{Buffer of ready computers}
\emph{Description:} Keep 6 pre-installed computers ready for use at OMT and Valcon, for emergencies or very quick recruitments.

\emph{Origin:}
In various interviews it was mentioned how this solution could ease the overall process, as it can deal with the situation of a critical recruitment.
(Sources: \quoteref{app:hanne_interview}{hanne_buffer} and \quoteref{app:jytte}{jytte_buffer})

\noindent \emph{Pros:} Would reduce response times to urgent recruitments.
Also useful if an employee's computer breaks, to reduce downtime.

\noindent \emph{Cons:} Needs a little more managing from IT, as
computers will need to be kept updated and ready.
As not all computers are in use, this creates a small financial overhead.

\emph{Verdict:}
Suitable. Having a buffer of computers would increase overall budget, but would solve many issues related to the recruitment process.

\subsection{Increase IT staff size}
\emph{Description:} Increase the IT staff size to increase throughput.

\emph{Origin:}
Baseline solution

\noindent \emph{Pros:} Makes IT capable of doing more.

\noindent \emph{Cons:} Expensive, and management increases.

\emph{Verdict:}
Not suitable. Simply increasing the size of the IT staff to accomodate demand is not a proper solution, as it would not increase the amount of capable workers. This is due to increased management of the IT department. The issues faced in the process cannot be reduced by simply having more employees working on it.

\subsection{New HR system}
\emph{Description:} Acquire a new off-the-shelf HR system to facilitate communication between accounting, IT, and recruiters.
Valcon is already looking at HR systems.

\emph{Origin:}
In the interview with accounting, it was mentioned how the current HR system was deprecated, and not very user-friendly.
(\quoteref{app:lisbeth}{lisbeth_HR})
A new HR system would help the process in a way, that information has to be typed in fewer times.

\noindent \emph{Pros:} Recruitment process may become more standardized.
Information becomes more consistent, as personal spreadsheets are less needed. 
This also safeguards against key employees leaving.
Errors are less likely, as number of manual entries of information are reduced.
It opens up the possibility of automating several key processes, reducing workload and wasted time further.

\noindent \emph{Cons:} Costly to implement, requires training of Accounting, IT, and Recruitment.
Also requires maintenance.

\emph{Verdict:}
Suitable. A new HR system would affect the company positively in relation to their business strategy.

\subsection{Help-program for finding initials and reading templates}
\emph{Description:} 
Write a small program that can read the template proposed above and enter the information into the IT systems.
Also able to propose initials for employees, and check whether the exchange mailbox has been used before.
Would reduce workload for IT and introduce fewer errors, but would require some maintenance and some time initially.

\emph{Origin:}
When interviewing the IT department it was mentioned how a tedious task it was to make up new initials. Since the making of initials are based on certain rules it is possible to generate them automatically. Furthermore they have to input data in various databases, as well as updating possible deprecated data.
(Sources: \quoteref{app:peter}{peter_initialer} and \quoteref{app:peter}{peter_old_employees})

\noindent \emph{Pros:} 
Less time used to making initials. Reduction of tedious tasks that must be conducted often.
Less time used on inputting data, reduction of possible typing errors.

\noindent \emph{Cons:} 
Generated initials might form unsuitable words, and the process has to be overridden by the employee.
System might be insufficient and not be capable of handling deprecated data correctly. 

\emph{Verdict:}
Partially suitable. The initials systems can be implement fairly easy. It can be taken into account that it must be possible to manually input initials if the generated initials are not sufficient.
The automated system will be too difficult to manage.

\subsection{Other small improvements}
\begin{itemize}
	\item Phone and internet paper in contract letter\\
	
			Put a phone and internet form in the contract letter, asking for the information the IT department needs.
			(Source: \quoteref{app:peter}{peter_blanket})			
	
	\item "Unkown department" in IT-systems\\
	
			Create two "Unknown Department" organisational units in the AD, one for Valcon and one for OMT, with relevant access rights. The IT department can then put employees whose department is unknown into these, and have a common place to look and update.
			(Source: \quoteref{app:peter}{peter_datamangel})
			
	\item Mail from OMT when recruitment is assured, but data is not yet available\\
	
			Define a process for OMT recruiters to alert IT when they know they will need new hires/subcontractors, so that IT can prepare before all data is gathered.
\end{itemize}