\label{app:lisbeth}
\begin{linenumbers*}
\subsection{Summary}
Lisbeth is an extremely important part of the hiring process. She is solely responsible for managing contracts for employees, and keeps a lot of data in private, local files. 
For Lisbeth, communication starts when either Hanne (Valcon recruitment), Jytte (OMT recruitment) or someone external (from departments outside of Denmark), writes an email with information on a new hire.
She manages the creation of contracts, maconomy accounts and salary systems.

From the interview, we found these key points:
\begin{itemize}
	\item{The old system was better for Lisbeth.}
		In the good old days, Valcon used Lotus Notes for everything. Lisbeth could automatically send mails from new employee data, and a lot of automisation was in place. During the years, more and more key functionality was removed from the Notes platform (invoicing was moved to Maconomy, mail was moved to Outlook, account management moved to Microsoft Active Directory). This slowly removed all the special functionality of Notes as an integrated solution.
	\item{Lisbeth has too much important information in her head.}
		 It is part of her job to have an overview of the pipeline of incoming and outgoing employees, and most of that is currently kept either in private excel sheets, or her head. This is not good for a business.
	\item{The hiring process for Lisbeth is not time consuming in itself.}  
		She uses 10-15 minutes per employee, best case. The problem arises when information is missing in mails from recruiters, as well as the inherent problems with copy-pasting information several times.
\end{itemize}


\subsection{Minutes by Jakob Vase}

Interview med Lisbeth
Introduktioner
9:30

[M:] Hvad sker der når en ny skal ansættes i Valcon?

[L: Jeg er] ansat til at lave løn og kontrakter
Når Hanne og Bjarke har fundet en medarbejder de vil ansætte skriver de til L[isbeth] med navn adresse og CV, hvornår han skal starte og hvilken afdeling.
Hvis hun er heldgi (H eller B [Valcon recruiters]) skriver [de] det selv ind i QHR. Så har L[isbeth] basic info. Notes understøtter ikke den organisation de har i dag, så [der] kan ikke tilknytte afdelinger/ løn / managers på "60\%" [af de nyansatte]. Når jeg får den lagt ind laver jeg en kontrakt. 
\linelabel{lisbeth_kontrakt}
Jeg skriver [sender] denne til kandidaten som underskriver kontrakten.
Så skriver jeg en mail til rekruttør at kontrakten er underskrevet. Går ind i Notes og ændrer kontrakten fra rekr.[utterings] process til ansat. Klikker på en knap så den laver skema til IT. Nogen gange har de [Hanne og Bjarne] ikke lagt stamdata ind, så skal jeg selv hente det.
\linelabel{lisbeth_IT}
Skriver nu selv en mail til IT om at de er kommet.
[Når de kommer fra OMT, skriver] Jytte fra OMT, [nogen] skal ansættes, navn, data, CPR, afdeling. Det er ikke altid jeg får lønne[n] at vide. [Jytte] Glemmer [nogen gange] start-dato [hvornår de starter med at arbejde]. Løs mail. Nogen gange er der kun 80\% af data.
Mail dialog om kontrakten [mellem Lisbeth og ?].
[Tanke- Meget snak om good old days (QHR ordnede meget selv, gav selv grundløn, alle indtastede derinde).]
\linelabel{lisbeth_vende_tilbage}
Nu skal jeg vende tilbage til rekruttør 99\% af gangene [og bede om mere data].
[I] 2009 var der i gang med en ny database, men den blwv stalled. Så kommer Outlook, og så dør Notes. Afdelinger, ledere etc. bliver ikke ajourført i Notes, så når jeg sender en mail [til ?] skriver jeg i headline på mailen ANSAT I ... MED CHEF ...
Man kunne søge på ting i QHR, hvilket man ikke kan længere. Se hvem der var under ansættelse, se hvem der var ansat. I gamle dage kunne man se hvad der lå i rekrutteringsdatabasen, og det kan man ikke i dag.

M[ichael siger] - Sværere at få et overblik?

\linelabel{lisbeth_ark}
L[isbeth svarer] - Ja
Hanne styrer det i et regneark, Lisbeth har også kontrakterne i regneark. Styrer det på en anden måde i dag. Hvis vi ikke er her er "fanden løs i laksegade". Der er ikke nogen udefrakommende som kan lave noget [i rekrutteringsprocessen]. Overleveringen til en ny medarbejder er stor. Folk der har været her længe og kender processerne.

[M:] Bliver problemet større?

[L:] Ja, organisationen bliver større / ændrer sig, og mindre og mindre passer i databasen. Der mangler information om OMT i systemerne. 
(Andet problem:)
Jeg kan ikke vide om en kandidat bliver til en ansættelse, så hvis han pludselig springer fra skal alt slettes. [Det] sker af og til.

[M:] - Skal det gå hurtigere end tidligere? Tænker på OMT, 4 dage...

[L:] Ja... Fra midt juli til nu kun 1 konsulent, 8 fra starten af November. Mette har ikke adgang til QHR (sekretær i OMT), ringer og spørger om hvem der starter i Nov, men det kan man ikke vide før måneden er forbi. Alle tror jeg har overblikket. Men glemmer jeg noget er det væk. Det står og hviler på mig.

[M:] Sker det at IT får noget at vide først?

[L:] Ja... En gang imellem tror jeg. Det kunne ske mere tidligere, men nu har jeg opdraget Indien [hvor problemet var størst]. Vi skal også styre alle underleverandørerne, de skal starte dagen efter, de er blevet lidt bedre, men det er stadig slemt. IT skal afvise dem og sende dem til mig. De må ikke have adgang til vore systemer, men de skal have en mail fra os og nogle initialer, så vi kan lægge dem på projekter og timeregistrere. Derfor skal de oprettes i Maconomy, de kan få adgang til vore systemer gennem en App, hvilket de godt må. [Om initialerne:] IT tildeler initialerne, for de har skøre hjerner. Det fungerer fint --- vi snakker lidt om om det skal være anderledes, vi kan få systemer som kan understøttes.

[M:] Det nye HR-system?

[L:] Mindkey har vi snakket med, intuitivt og virker godt. De skal forstå vores case. [De skal ikke vende tilbage med et] hov der vra lige noget. Til at gå til og prisen er ikke vanvittig. Så må vi se. Der er jo det at man kigger på økon. fee hvert år, hvor mange brugere, alle skal ikke have adgang, forhandle en anden pris for mindre adgang. Jytte/Hanne/sudo/balla/Pelle/Mat [sudo og balla er indiske ansatte, Pelle og Mat ved jeg ikke hvem er] skal have fuld adgang, og kan ligge ind i systemet. Jo kortere vej jo rigtigere oplysninger, [for eksempel] Jakob med K eller C. De skal så køre hele processen i systemet, så de får meget hjælp, og hvis der sker nogen noget så ligger de der.

[M:] Det nye skal være ligesom QHR var engang?

[L:] Ja. Samspillet skal op og køre igen. I mange andre virksomheder rekvirerer de [sådan her:] "så rejser der en, så kommer der en ny". Vi har ikke ledige stillinger. Ikke på konsulentsiden, måske på backoffice. Hvis den rigtige ikke er der ansætter vi ikke nogen. På den måde er vi anderledes ([det er et] problem i forhold til det nye system). Løn er salgsmål og faktureringsmål. Der er ikke lønklasser, det er individuelt. Erfaringsmæssigt forskellig løn, og det skal systemet understøtte. Det skal ligesom skræddersyes til os. Vi har ikke holdt opfølgningsmøde, men Mindkey var det bedste af 3-4, der er masser af HR-systemer. Ofte er der kun rekruttering eller noget andet.

[M:] Valcons udfordring er at deres organisation er så anderledes?

[L:] Ja, det nye system er Microsoft baseret, så de skulle kunne snakke sammen. [Her snakker hun om problemet med at systemerne ikke kan snakke sammen. Specifikt Maconomy og Lotus Notes.]
Jeg har også et lønsystem.
[Beskriv hele processen:] Jytte skriver mig en mail, jeg tager oplysningerne og lægger dem i QHR, så taster jeg det ind i Maconomy, og til sidst i lønsystemet. 4 manuelle indtastninger. Man kunne i hvert fald få et step væk. Det danske lønsystem (bluegarden), norsk revisor og svensk system. Revisor i Canada, engelske lønninger hos Dorthe. Indien og Kina klarer sig selv. Alle skal igennem mig, for alle skal have adgang til medarbejderne i Maconomy.

[M:] Hvor meget tid bruger du på ansættelser?

[L:] Det fylder meget. Det er manuelt det hele.

[M:] Hvad med hvis det hele er der i QHR? 

[L:] 15 minutter for kontrakten. Maconomy 5 minutter ca. Samme i lønsystemet. Problemet er når der skal snakkes sammen meget. Nogen gange sidder man og roder lidt[, finder på småting for at få det til at hænge sammen]. Finpudsning er der ikke plads til. Hvis man havde valgt Maconomy til medarbejderstyring, kunne det være nemmere, men det var dårligt så det blev valgt fra. Den er død[, Maconomy skal ikke styre medarbejdere]. Der skal vedligeholdes på medarbejdere. Mindkey skulle også kunne udstyr. Vi ser på det hele. Ansættelse, vedligeholdelse, afsked. Hvorfor siger medarbejdere op, det skal også kunne blive i systemet. Hvorfor smider vi selv folk ud? [Det skal også være i systemet.]

[M:] Skal du give grønt lys?

[L:] Ja, når jeg trykker ansat får IT det at vide. Jeg er ikke interessert i Hannes database, og de skal ikke kunne se hvad der sker i OMT eller hos mig.

[M:] Er der også snak om opdelign i det nye system?

[L:] Ja.
Alle bør have adgang til systemet, forskellige brugerflader. Hvis jeg har været på et kursus skal jeg selv lægge det ind i systemet, så kan andre se det. Det sker allerede lidt med CV, men der står alt ikke og bliver ikke opdateret.
Jeg ved fra andre firmaer at det kan spille sammen det hele.
Maconomy er fast [det er godt til det det skal], men har svært ved at snakke med andre systemer. Sap er meget dyrt. De var her.
Maria og jeg var på messe sidste år i Köln, på HR messe, vi snakkede med nogle SAP folk, og der var det interessant, men dem der sælger det herhjemme var ikke så interessante. Vi vil bare have det hele, og vi ville helst have Notes, for der virkede det hele. Men det gør det ikke længere. maconomy slog det ihjel 2009. Derfra har processerne ikke været understøttet optimalt. 

[M: Bliver problemet større på grund af væksten? 

L:] Ja væksten, men det bliver sårbart, for det er personafhængigt. Der ligger processer som kun jeg kender til.

[M:] Det vil det nye system hjælpe på?

[L:] Ja, men der vil altid være masser af manuelle indtastninger. Der er ikke EN der elsker maconomy. Vi syns heller ikke det er skidegodt. Det er blevet opdateret, men det er dyrt og det bliver måske ikke bedre udadtil.
[Det nyesystem:]
Konsulernter skal ikke ned i det, det skal være et usynligt system. Jeg vil gerne have et meget lækkert system. [Men jeg ved godt det ikke kommer til at ske.]

[M:] Hvem skal opdatere data?

[L:] Jeg får det at [v]ide, og så fortæller jeg det il IT. Så jeg retter til i Maconomy og IT i AD. [Jeg] kender ikke til AD.
\end{linenumbers*}