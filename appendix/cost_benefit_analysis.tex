\label{app:cost_benefit_analysis}
In this appendix, we look at the costs of each of our proposed solutions and write a few benefits.
We will not look at the new HR system, as Valcon is already working on it and we do not know enough about the system to correctly estimate costs and benefits.
To see a description of the solutions, see appendix \ref{app:solution_propositions}

\subsubsection{Software that does it all}
\begin{itemize}
\item \emph{Costs} - Very high. (\textgreater 1,000,000 DKK, drawing on comparable IT solutions in other companies.)
\item \emph{Benefits} - Accounting and IT would be happy, as all data would have to only be entered once. But consultants would be less so, as their work would be slowed down by data entering and a heavy program. (Lower income.)
\item \emph{Conclusion} - This investment would never break even.
\end{itemize}

\subsubsection{Template for recruitment}
\begin{itemize}
\item \emph{Costs} - Three hours for creating the template, an hour every six months to keep it up to date. (500 * 3 = 1500 DKK initially, 500 * 2 = 1000 DKK annually. (Assuming 500 DKK/hour for the employee making and maintaining the template.))
\item \emph{Benefits} - Lisbeth saves time, as data isn't missing, but Hanne and Jytte spend a little more time gathering the information. All three save time as less communication between them is needed. (Lisbeth saves appr. 10 minutes per recruitment, Hanne goes even, Jytte goes even.) (Minutes saved * number of recruitals annually / 60 = hours saved annually = 10 * 180 / 60 = 30 hours saved annually = 9000 DKK saved annually (assuming 300 DKK/hour))
\item \emph{Conclusion} - This investment would break even in the first year, even saving 6500 DKK, and would save 8000 DKK annually afterwards.
\end{itemize}

\subsubsection{Change of attitude toward IT}
\begin{itemize}
\item \emph{Costs} - (Example:) A small, internal campaign to change company attitude might cost appr. 20,000 DKK (Two days work for a designer, printing posters, sending emails, lost time for consultants).
\item \emph{Benefits} - Better communication with IT. We cannot see any financial gain. Would reduce frustration in IT and make IT feel more appreciated.
\item \emph{Conclusion} - Won't break even, but the non-financial benefits are important.
\end{itemize}

\subsubsection{Buffer of ready computers}
\begin{itemize}
	\item \emph{Costs} - 240,000 DKK initially (20,000 DKK * 12 computers). Appr. 1,200 DKK annually for maintenance (8 hours work for a student-helper, 150 DKK per hour pay.)
	\item \emph{Benefits} - Shorter response times from IT, less frustration in IT. Time saved in IT as the computers can be created at the same time. 96 hours saved by student helpers annually = 14,400 DKK saved annually. In cases where a quickly hired employee would not normally be able to start work, because of a delay in getting them a computer, having a buffer can potentially save a lot of money.
	\item \emph{Conclusion} - Will break even after less than 18 years, potentially faster if there are a couple of cases of lost workdays per year. Less frustration in IT and better response times make it worth it.
\end{itemize}

\subsubsection{Increase IT staff size}
\begin{itemize}
	\item \emph{Costs} - Appr. 1,000,000 DKK annually. (Full time employee + management and resource costs.)
	\item \emph{Benefits} - Less frustration in IT. IT would be able to handle a higher workload generally, not just with regards to the process of new employee registration.
	\item \emph{Conclusion} - This is a very expensive solution to the problem. 
\end{itemize}

\subsubsection{New HR-system}
\begin{itemize}
	\item \emph{Costs} - Because the Valcon Group is already looking into a new HR-system it doesn't seem necessary to look at the costs.
	\item \emph{Benefits} - Assuming that the Recruitment, Accounting and IT departments would all be able to access shared information within the new system a total of approx. 60 minutes can be saved per hire across departments. With about 150 new hires per year, a total of 150 hours can be saved. Assuming an average cost of 500 DKK per work hour this is equivalent to 75,000 DKK. (Sources: Every department has specified their amount of work used in e-mails, seen in appendix \ref{app:emails})
	\item \emph{Conclusion} - Even though the focus of a new HR-system isn't solely the recruitment process, it would also save a considerable amount of money with regards to the recruitment process.
\end{itemize}

\subsubsection{Help-program for finding initials and reading templates}
\begin{itemize}
	\item \emph{Costs} - Writing the program and maintaining it will cost a programmer 5 days initially, and then 1-2 days annually. (12000 DKK initially, then 4800 DKK annually, assuming 300 DKK / hour.)
	\item \emph{Benefits} - Less time spent on entering data, less errors, less frustration for IT. (5 minutes * 156 recruitments / 60 * 300 DKK /hour = 3900 saved annually)
	\item \emph{Conclusion} - This won't break even. Should only be implemented to make IT employees happier (which might be worth it).
\end{itemize}

\subsubsection{Other small improvements}
These are described in detail in appendix \ref{app:solution_propositions}.
\begin{itemize}
	\item \emph{Costs} - These are all free or have very low costs (below 500 DKK over 5 years)
	\item \emph{Benefits} - The benefits are mainly time saved for IT and accounting, except in the case of the mail from OMT, which would help reduce frustration.
	\item \emph{Conclusion} - All three of these save a little money or increase work enjoyment slightly for IT and/or accounting.
\end{itemize}