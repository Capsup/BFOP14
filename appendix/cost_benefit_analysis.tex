\label{app:cost_benefit_analysis}
These calculations are based on the recruitment and resignation data sheet in appendix \ref{app:recruitment_data}.

\section{Standard process and edge cases}
\begin{itemize}
\item In 2012/13, 118 persons were hired or subcontracted.
\item In 2013/14, 154 persons were hired or subcontracted.
\item In 2014/15, 80 persons have been hired or subcontracted so far.
\end{itemize}
These numbers show how many computers have been made ready by year.
\emph{There is a small increase in the number of computers being readied (30\% in 2013 and (projected) 4\% in 2014).}

\subsubsection{Time spent on computers and data entering}
Peter gave an estimate for how long it took him to set up a computer and enter data into the system, 45 minutes (appendix \quoteref{app:peter}{peter_estimat_setup}). (We realize Peter may be wrong about this. We haven't been able to observe the process, but we still think the calculations are useful.)

This means, that 88.5 hours have been used in 2012/13, 115.5 hours in 2013/14, and 60 hours in 2014/15 so far on computer setup and data entering.

With less than 150 hours used every year, even if we save 15 minutes for every recruitment, this is only 50 hours saved every year, so \emph{our proposed solutions need to be inexpensive.}

\subsubsection{Time spent on other things related to recruitment}
Peter uses time on gathering information on telephone and internet connection preferences from the new employees, appr. 10 minutes on average.
(appendix \quoteref{app:peter}{peter_estimat_contact})
This is only on the Danish recruitments, and not on subcontractors.

Furthermore, he uses time on Danish resignations, both subcontractors and hires, appr. 15 minutes on average.
\todo{Where do we know this from}
(We realize Peter may be wrong about these. We haven't been able to observe the process, but we still think the calculations are useful.)

\subsubsection{Total time spent}
Here follows the total amount of time spent on recruitments and resignations each year (barring edge cases).
\begin{itemize}
\item 2012/13 - 111 hours.
\item 2013/14 - 148 hours.
\item 2014/15 - 75 hours so far. (Half the year has passed, so a reasonable estimate is that he will use 150 hours total.)
\end{itemize}

As long as there are few edge cases, \emph{the time spent is a small problem.}

\subsubsection{Edge cases}
We haven't been able to get any numbers on the amount of edge cases in the process.

This is problematic, as our main focus is documenting that there is a problem, and the amount of time used without edge cases is small.

And even when edge cases happen, as the process doesn't change, \emph{the time spent doesn't increase, only the amount of frustration in the IT department.}

\subsubsection{Benefits to improving the process}
There are some benefits to changing the recruitment setup process.

Currently, IT and accounting are both frustrated by the process.
By improving the process, the work enjoyment could improve.

Finally, improving the process can speed up response times on support and recruitment from IT and on recruitment in accounting.

\subsubsection{Conclusion of process analysis}
There is little financial gain in changing the recruitment setup process.

The benefits, however, are important and in line with the Valcon's employee values.

So the solutions we propose must be without or with very little cost.

\section{Cost/benefit of solutions}
In this section, we look at the costs of each of our proposed solutions and write a few benefits.
We will not look at the new HR system, as Valcon is already working on it and we do not know enough about the system to correctly estimate costs and benefits. \todo{maybe remove}
To see a description of the solutions, see appendix \ref{app:solutions}

\subsubsection{Software that does it all}
\begin{itemize}
\item \emph{Costs} - Very high. (\textgreater 1.000.000 DKK, drawing on comparable IT solutions in other companies.)
\item \emph{Benefits} - Accounting and IT would be happy, as all data would have to only be entered once. But consultants would be less so, as their work would be slowed down by data entering and a heavy program. (Lower income.)
\item \emph{Conclusion} - This investment would never break even.
\end{itemize}

\subsubsection{Template for recruitment}
\begin{itemize}
\item \emph{Costs} - Three hours for creating the template, an hour every six months to keep it up to date. (500 * 3 = 1500 DKK initially, 500 * 2 = 1000 DKK anually. (Assuming 500 DKK/hour for the employee making and maintaining the template.))
\item \emph{Benefits} - Lisbeth saves time, as data isn't missing, but Hanne and Jytte spend a little more time gathering the information. All three save time as less communication between them is needed. (Lisbeth saves appr. 10 minutes per recruitment, Hanne goes even, Jytte goes even.) (Minutes saved * number of recruitals anually / 60 = hours saved anually = 10 * 180 / 60 = 30 hours saved anually = 9000 DKK saved anually (assuming 300 DKK/hour))
\item \emph{Conclusion} - This investment would break even in the first year, even saving 6500 DKK, and would save 8000 DKK anually afterwards.
\end{itemize}

\subsubsection{Change of attitude toward IT}
\begin{itemize}
\item \emph{Costs} - (Example:) A small, internal campaign to change company attitude might cost appr. 20.000 DKK (Two days work for a designer, printing posters, sending emails, lost time for consultants).
\item \emph{Benefits} - Better communication with IT. We cannot see any financial gain. Would reduce frustration in IT and make IT feel more appreciated.
\item \emph{Conclusion} - Won't break even, but the non-financial benefits are important.
\end{itemize}

\subsubsection{Buffer of ready computers}
\begin{itemize}
	\item \emph{Costs} - 240.000 DKK initially (20.000 DKK * 12 computers). Appr. 1.200 DKK anually for maintenance (8 hours work for a student-helper, 150 DKK per hour pay.)
	\item \emph{Benefits} - Shorter response times from IT, less frustration in IT. Time saved in IT as the computers can be created at the same time. (New employees can start work earlier in quick cases means money saved) (96 hours saved by student helpers anually = 14.400 DKK saved anually, maybe some more from new employees starting early.)\todo{Computer repair time saved?}
	\item \emph{Conclusion} - Will break even after less than 18 years without taking new recruitments starting early into account. Less frustration in IT and better response times make it worth it. \todo{net present value?}
\end{itemize}

\subsubsection{Increase IT staff size}
\begin{itemize}
	\item \emph{Costs} - Appr. 1.000.000 DKK anually. (Full time employee + management and resource costs.)
	\item \emph{Benefits} - Less frustration in IT.
	\item \emph{Conclusion} - This is a very expensive solution to the problem.
\end{itemize}

\subsubsection{New HR-system}
We won't describe this here, as Valcon is already working on it.\todo{Write this!}

\subsubsection{Help-program for finding initials and reading templates}
\begin{itemize}
	\item \emph{Costs} - Writing the program and maintaining it will cost a programmer 5 days initially, and then 1-2 days anually. (12000 DKK initially, then 4800 DKK anually, assuming 300 DKK / hour.)
	\item \emph{Benefits} - Less time spent on entering data, less errors, less frustration for IT. (5 minutes * 156 recruitments / 60 * 300 DKK /hour = 3900 saved anually)
	\item \emph{Conclusion} - This won't break even. Should only be implemented to make IT employees happier (which might be worth it).
\end{itemize}

\subsubsection{Other small improvements}
These are described in detail in appendix \ref{app:solution_propositions}.
\begin{itemize}
	\item \emph{Costs} - These are all free or have very low costs (below 500 DKK over 5 years)
	\item \emph{Benefits} - The benefits are mainly time saved for IT and accounting, except in the case of the mail from OMT, which would help reduce frustration.
	\item \emph{Conclusion} - All three of these save a little money or increase work enjoyment slightly for IT and/or accounting.
\end{itemize}