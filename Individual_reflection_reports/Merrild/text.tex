\section*{Project goals}
Already at our initial meeting with Danni Jensen, head of IT at Valcon, he told us that his goal for the project was that he could document that the perceived problem existed.
Additionally we told him that we would only conduct the design project, and would not be implementing any of the solutions we would come up with.
I believe these two facts combined have had a hindering effect on our project.

The investment from Valcon has not been as high as it would have been if their goal was to use our proposed solutions to improve upon the problem.
Similarly our investment would possibly have been higher if we knew that our solutions might be implemented.

In relation to the MUST principles I think that particularly the \textbf{genuine user participation} and the \textbf{anchoring visions} principles have suffered.

\section*{Genuine user participation}
While we applied some ethnographic methods in the way we conducted our interviews with the employees at Valcon it would be hard pressed to call it proper \textbf{genuine user participation}.
According to Bødker et al. \textbf{genuine user participation} \myquote{calls for the active participation ... by representatives of staff members}.\footnote{Bødker et al., 2004, page 58, first paragraph.}

In order to better realise this principle we could have involved the employees more in our process.
Firstly, in the in-depth phase, after having analysed the first interviews, we could have brought our results to the employees for verification.
This would have ensured that our understanding of the work process was in line with the employees' understanding.
Secondly, and more importantly, we could have involved the employees more in our innovation phase.
As it stands all of our solutions have been proposed without any involvement of the employees at Valcon.

In an ideal world we could have conducted workshops, bringing together employees from the various departments, in order to better ensure that the proposed solutions fit with the current organisation.
Such workshops could also function as a way of \textbf{anchoring visions}.

\section*{Anchoring visions}
As it was clear from the beginning of the project that the proposals set forth would not be implemented we did not focus on \textbf{anchoring visions}.
Bødker et al. state, that ``\emph{In the process}[of anchoring visions]\emph{, the design team is attempting to gain wider support for its proposal}''.\footnote{Bødker et al., 2004, page 70, last paragraph.}
This is in direct contrast to the above fact, so it is unsurprising that we overlooked this principle of the MUST method.
I can however see good reason for keeping this principle in mind when conducting IT design projects with higher impact.

\section*{Lack of project impact}
When conducting an IT design project during a 14 week course the scope of the project is by necessity quite small.
This together with the aforementioned project goal from the company has made it more difficult for me to invest in the project and properly apply the MUST method. 
I would have liked to work with a project which had a bigger impact.
In particular I would have wished for the project to be more closely related Valcon's business.

\section*{Lack of business focus}
Since the problem that we have been investigating is related to the internal support functions at Valcon it has been difficult to relate it directly to the business.
It is always more difficult to measure the exact benefit of a support department within a company, which is purely an expense, versus the departments which generate revenue.

This caused us to keep our focus on Valcon's business at a minimum.
When we were told that it wouldn't be possible to retrieve their business strategy we didn't put a lot of effort into otherwise deducing it.
In hindsight we could have spent more time during our in-line phase trying to extract more information through better formulated questions.
If we had done this we could possibly have conducted a more thorough cost/benefit analysis.

\section*{Conclusion}
Our project might not have been conducted perfectly, but that is to be expected as it is a learning process.
As such I think the project has been a success.
It has allowed us to apply some of the methods which we were taught during the course to a "real life" problem.
Even more importantly it has allowed us to reflect upon this process, giving us a better understanding of the methods and tools taught than if we had merely read/heard about them.
That being said, I think we could have learned even more had we conducted a project with a more direct impact on the business of the company for whom we conducted it. 
This would also naturally induce a higher investment from the company.