\subsubsection{Introduction}
%\todo{< Introduction to the topic (Michael in the group and implications)  1/4 page >}

This reflective report will describe some of the implications encountered during the project, in regards to having a member of the group being an employee of the company under observation. This group member is namely Michael (mifr).

%\todo{< Theory aspect of the topic (Genuine User Participation, Tacid Knowledge, perceived knowledge and say/do insight) 1/4 page >}

From an initial analysis some of these implications were relatable to relevant course theory, namely genuine user participation, tacit knowledge, perceived knowledge and the say/do theory.
Through this report i will explain each theory, the implication associated with it and what could have been done to avoid it.
This will conclude in a reflection of our overall experience, having a group member being a part of the company under observation.

\subsubsection{Genuine User Participation}
%\todo{< Explain Genuine User Participation and relate it to topic 1/4 page >}

The theory of Genuine User Participation is rooted in that the project group must be in direct collaboration with the key persons under observation in the company, in order to reflect the visions of the company, instead of those of the project group itself.

Throughout the project we visited the company and interviewed the participants, in order to draw opinions and knowledge as per standard process. However, later in the project we got accustomed to ask Michael about uncovered areas which required further explanation. This created a gap between the project group and the company under observation, since most communication "with the company" now went towards Michael, rather than the actual company participants. This resulted in some of the visions generated from the collaboration to become those of Michaels perception rather than that of the actual company participants.

Conducting a participatory IT design project relies heavily on being in close collaboration with the company under observation. Company visits should not only be to acquire data, but also to uphold the genuine user participation. To avoid this implication we, as a project group, should have had planned meetings to follow up on the knowledge we had acquired to verify our interpretation of it, rather than verify it through Michael. This would lead to more correct knowledge, as well as better genuine user participation.

\subsubsection{Tacit Knowledge}
%\todo{< Explain Tacid Knowledge and relate it to topic 1/4 page >}

The theory of Tacit Knowledge is that to individual people, some knowledge is regarded as common knowledge and is thus not explained, when portraying their understanding of a concept.

As it was explained in the previous section, we used Michael as a way of verifying our acquired data. Towards the end of the project, we realised how this had complicated the process, since some information was incorrect or insufficient, as it was tacit knowledge for Michael.

Since the communication with Michael was very informal and unplanned, we forgot to apply the theories of data acquisition to the information retrieved thereby, and thus we never considered to remain critical towards what was conveyed. By this, we did not take tacit knowledge into consideration, as we would normally have from an interview.

This implication could have been avoided by assuring that all channels of new knowledge were subject to general knowledge acquisition theories, which takes into consideration that knowledge is individual and should therefore be analysed critically.

\subsubsection{Perceived Knowledge}
%\todo{< Explain perceived knowledge and relate it to topic 1/4 page >}

The theory of perceived knowledge is that knowledge exists in two forms, the knowledge as it is known by the individual relaying it, and the knowledge as it is perceived by the individual receiving it.

When performing data acquisition throughout the project, we tried to be critical and regard how we subjectively understood the knowledge that was objectively relayed. This helped us to further understand the actual data which we could rely on.

As explained previously, we forgot to apply the theories of data acquisition, when communicating with Michael, and thus we did not question whether the knowledge that Michael relayed was perceived wrongly by us, nor if it was perceived wrongly by Michael initially.

The solution to this implication is the same as described previously, in that we should always apply our knowledge acquisition theories to all forms of retrieved knowledge.

\subsubsection{Say/Do}
%\todo{< Explain say/do and relate it to topic 1/4 page >}

The say/do theory is that when a participant explains a process, they say one thing and do another. This is mainly because most people know how they are supposed to conduct their work based on the planned process, but disregards this under actual circumstances.

Normally this is circumvented through observation of the process, however due to the nature of the process we analysed, this was not possible. In this case it was relevant to have Michael as member of the project group, as he would, in most cases, know how the tasks were actually conducted, and so we could still manage the implications of the say/do problem.

The explanation given by Michael in this context should however still be subject to knowledge acquisition theories as explained previously.

\subsubsection{Reflection}
%\todo{< Reflect on topic and implication 2/4 page >}

Having a project group member from the company has a great impact in regards to how the company is perceived. From the theories that has been described and put in relation to our project, we can see how these implications are both positive and negative in regards to different aspects of the project process.

In some participatory IT design projects, it is custom to have a member of the company in the project committee, serving the same purpose as what we used Michael for, namely a member that is closely related with the company, that, for instance, can follow up on acquired knowledge and data.

The reason for our implications was that we did not handle Michael as this external company group member, since he was a part of the core study group, and thus we did not apply relevant theories that are used to circumvent the issues we faced.

In retrospect it is important to not only remember the course theories, but especially when to apply them. We used them whenever we saw fit, in a project relational situation, and thus we used them as "by the book", rather than by what the situation demanded.

This is something to remember for future projects, as the situations in which ethnographical theories should be used, will mostly be identifiable through experience.

\subsubsection{Conclusion}

Having a member which is part of a company is not a hindrance of conducting a good project. Some projects even relies on having a member of the project committee being part of the company.
However it is just as important to remain critical towards knowledge acquired through the member, as it is with all other knowledge generated through interviews and observations.
In general, there is no such thing as "pure" knowledge, which can simply regarded as correct, and thus new knowledge should always be a subject to ethnographic studies, if they are to be applied to the project.

%< Total 8 / 4 = 2 pages >