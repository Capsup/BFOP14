\chapter*{Reflection report}

The subject of this individual reflection report will be to highlight what I found great, as well as what I found lacking in our project. The report will be split into relevant sections, highlighting first my experience being an employee writing about his own work practice, and then what I found lacking in our observations. Finally, there will be a section with some last remarks about the project as a whole.

\section*{My position in Valcon - A comment on Tacit Knowledge and Genuine User Participation}
I am a part time student help at Valcon, and have been with the company for 2.5 years. As a student help in the IT department, I know a great deal about the IT processes of the company. I also believe that I have at least a fundamental understanding of the company values, the work processes of the consultants and some of the politics associated with the relationship between consultants and the IT department / back office.

\subsubsection{Tacit Knowledge}
Tacit knowledge is the "hidden" things that are regarded as common knowledge, or just not thought about, and therefore is not explained when relaying the understanding of a process.

When we started out, I did not realise how big an impact my knowledge would have on the project. Every time one of the group members had a question about Valcon, I was the go-to guy, and my answers were treated as fact. While we did take care not to treat my information as a direct source, and nothing that we wrote in the report is taken directly from what I said, it shaped the course of the project in a major way, as we could structure a lot of our interview questions from the knowledge I already had.
As much as I took care not to say anything I was not sure of, I am not in doubt that we presume a lot of things that may be just presumed knowledge on my part. \\

I was surprised to see just how much of my knowledge on relevant processes had to be practically pried out of me. Many times, a group member asked me a quick question about a specific edge case, and my response was often that there was a specific thing we do in that case. I never thought up these edge cases by myself, as I didn't think of their relevance to the project the way an outsider did. It was necessary for the others to identify the process, as I was too used to it being the norm, not asking questions. I use this example as a clear case that you can read all you want about tacit knowledge, you still won't be able to avoid it in yourself.

\subsubsection{Genuine User Participation}
Genuine user participation is the idea that a project group should be in direct collaboration with key persons from the company. This leads to the company's values and visions being reflected in the analysis, and not the project groups.

I believe that this project is a strong case for having at least one person from the company be part of the project group. Furthermore, it is clear that conducting a project like this exclusively with employees would be a bad idea. Maybe it's because I'm afraid of repercussions from questioning the system, maybe it's because I don't have a problem with the existing notion that consultants shouldn't spend any unnecessary time on IT, but several times I've found myself cringing slightly whenever a group member spoke of changing work practices I do not believe are wrong, such as changing the consultants' status.
And I believe that these frustrations from the other group members are healthy for the project, and they *may* even be a sign that the current system could be better - I just don't see it, being so used to the current ways. My point is, in a group composed entirely of student helps from Valcon IT, a lot of the things we debated in the group would never have been questioned.


\section*{Lack of observations and proper cost/benefit}
\subsubsection{Perceived knowledge is dangerous}
In hindsight, we should have spent more energy trying to find a way to observe the processes relating to our project. Yes, they happen periodically and would be hard to get to observe as they happened, but we could still have made Peter and Lisbeth especially guide us through their processes on a test case. Most of our report is based solely on things we have been told in interviews, and not things we have observed happen. Perceived knowledge is different from actual knowledge, and we've been working solely with what is potentially perceived knowledge.

We might have identified further opportunities for improvement if we had observed the work practices involved, instead of relying on accounts from interviews.

%\subsubsection{Cost/benefit is tough when no money is involved}
%As it turned out, the problem was never about saving money. We didn't %realise this when we started out, and we could have guided our interviews %in potentially more beneficial directions had we seen it from the start.
%I'm not saying it was impossible, just that we shouldn't have waited so %long with getting that employee data and finding out that it isn't about %saving time as much as it is about reducing frustration and error counts - %especially since we have no error counts to calculate from, and didn't %really try obtaining any (again maybe a fault of mine, as I claim to know %that the data doesn't exist, so we never asked).

\section*{OMT and the dangers of expanding your scope}
\subsubsection{Realising you are looking at the wrong company}
As we conducted our interviews it became more and more apparent that the problems we would be able to do the most about originated at OMT. There were some talks in the group of going there and making a whole other in-line and in-depth on them, and I am glad that we didn't. If we had expanded our scope that majorly, we would probably not have had the time to make our final document polished. 
It is a valuable lesson to me that you can do a lot of the right things, and your conclusion can still be "we should look over there, because we can't do much here". The real world is not like most school assignments, with the best solution hidden in the scope you have been given. 

\section*{Last remarks}
Overall, I am satisfied with the project. There are a lot of things I would have liked to do differently, but it has been an interesting experience, and I have taken a lot away from it, especially the realisation that you can't look at your own work domain as an outsider, no matter how much you try to forget what you usually do.