\subsubsection{Introduction}
This reflective report will look at the In-Line Analysis Phase of the project and how working with the Valcon Group has shaped the project and my personal experiences with it.

From the very beginning of the project we were told by Danni, the IT manager of Valcon, that it would be impossible for us to get a document explaining Valcon’s business and IT strategy. This put us in a position where we had to try and figure it out as best as possible on our own, for us to be able to align the project with the business and it goals of Valcon.
However, Danni decided to change his opinion of this later in the process and gave us a description of the IT strategy so that it was easier for us to perform the strategic alignment.

\subsubsection{Strategic alignment}
During the course of the in-line analysis phase, the primary goal  is to first and foremost clarify the goals of the project, according to the company’s business and IT strategies. This is done to clarify possible work domains that can later be studied further for integrating IT. The project group examines the company and the environment within which it operates and attempts to construct a business canvas for the company to get an overview of the company as a whole, making it easier to figure out what work domains the company might be interested in focusing on.

During our project, as mentioned above, we were told it was impossible for us to acquire neither the IT strategy nor the business strategy. So to get an idea of what kind of business strategy Valcon employed, we designed our initial interviews such that we asked the various employees of the company about the business goals of Valcon. Taking this information and trying to extract as much information as possible from Valcon’s website, we tried to come up with a full picture of Valcon and their business strategy, leading us to the work domains that we’ve explained in full in the report.
The same thing wasn’t possible with the IT strategy, since there was no way for us to figure out this information on our own. Luckily, during a later meeting with Danni, he provided us with a detailed overview of Valcon’s IT strategy such that we could properly incorporate the strategy into the rest of the project.

What surprised me the most with the strategic alignment of our project was the fact that during the initiation phase of the project, we quickly came to the conclusion that there wasn’t much need for in line analysis in the first place and we could identify ourselves with situation 2 as explained in the book “Partipatory IT Design” at page 120. However, as we slowly moved towards the in-depth analysis part of the project, as we dug deeper into the problem and uncovered more information, we slowly realised that we might have to look at the strategic alignment of the project again.
We didn’t exactly go back into the in-line analysis phase, but rather we worked in between the in-line analysis phase and the in-depth analysis phase during most of the project. When we finally got the IT strategy laid out to us, we could then move into the in-depth phase completely and work out the work domains for us to investigate further.

\subsubsection{Work domains}
Work domains are the actual results of the in-line analysis phase. It’s the parts of the process that the project group determines are especially interesting to dig deeper into during the in-depth phase. 

At Valcon, we were told that we could get to speak with almost everyone in the company as long as Danni was told beforehand, so that he could let the employee know before we contacted them. During the early initation phase we realised that it would be interesting to talk to the higher-ups in the company, namely the ones who had the responsibility for the entire recruitment process, to get their view on it and hear that they had to say. However, we were told by Danni that this wasn’t possible and this proved to be a bigger problem later on that we realised at first. 
Talking with the employees doing the recruitment and setting up the IT wasn’t sufficient, as there were times where we ran into the problem where the process was as it is because that’s just what the higher-ups had said, but the employees were annoyed with how it is. The same problem arose since we didn’t have the opportunity to go to OMT in Odense, as it was very relevant for us to look at the process from their point of view as it turned out to be at OMT where the problems primarily arose.

I didn’t think this would be a problem at first, but not having free access to talk to all interesting parties in the company proved to be a bigger pain than originally expected. This was due to the fact that since the higher-ups were the ones who had all the power to directly impose changes on the process, it was the employees who ran into the daily problems with the process. Not being able to hear it from the perspective of those who made the process gave us a skewed idea of why things were as they were, which could eventually result in us choosing the wrong work domains to further investigate due to wrong information.

\subsubsection{Reflection}
One of the things we should have been better at during the project was getting actual data from Valcon, which we could then have used to backup the information that we got from the interviews. There was an interesting group meeting, where we had finally received some data from the HR department about how may new employees and subcontractors they hired and how many left the firm, over a span of two and a half years. We looked over this data while started pulling out the interesting bits that we needed for cost benefit analysis and started discussing the numbers and what they meant for the project. It wasn't until this point, that we fully realised just how small a problem we were working with in terms of money and man hours. 

Until this point, we were working under the assumption that there was a lot of man hours spent in this process, but when we looked at the actual data, it clearly proved there wasn't much time spent at all. That's when we realised that the major issues the employees were talking about during the interviews, only happened whenever the process had exceptions. But since this was during late November, it was too late for us to start gathering the data that proved that many exceptions occurred during the process, so it was extremely hard to work this into the cost benefit analysis since we had no idea how often it happened.

Elizabeth did mention to us during a class at one point, that it was important for us to figure out what kind of data would be relevant for the project early on, such that it was available at the end of the project when the report had to be done. At that point, we should probably have sat down and looked at the data that would be relevant for our project.

\subsubsection{Conclusion}
Even though we ran into several issues during the project, namely inaccessible employees for interviewing and problems with getting the official company’s business goals, we managed to get most of the necessary information for us to figure out what we believe are the correct work domains to work further in depth with.