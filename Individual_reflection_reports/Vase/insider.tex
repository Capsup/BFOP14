\subsubsection{Working with an insider}
One of our group members, Michael, works at Valcon, and was there twice a week during the project.
This had both positive and negative effects.

It meant that we had a person at the company twice per week, to ask small questions, chat with his colleagues about the project, and so forth.
It also meant we had access to internal systems, the mailing lists, employee database, TechAdm, chat and others, which we then did not have to get permission to use for our project.
(Maybe we should have asked permission anyway. That would have been more political.)
Finally, it helped in that we could just ask him when faced with interview answers we didn't understand or some information was missing, saving precious time.

The first and third benefits also had negative effects.
When he was at the company twice per week, there was little reason for the rest of the group to go after the first two visiting days.
Without him, we would have definitely gone a third time, to the benefit of the analysis.
The third benefit had a hidden side effect.
As we got used to just asking Michael about things we didn't understand, we all ended up sharing his understanding of the process, which might not be correct.
As Bødker (2004) writes on page 71, we developed 'insider reasoning', letting Michael's opinions affect the project more than I would have liked.
We discussed this, but decided that the gain in time was worth the loss in depth.

In future projects, I will do my best to avoid insiders in the project group.