\subsubsection{The MUST principles and my experiences}
The MUST method builds on 4 core principles:
\begin{itemize}
\item Coherent vision for change
\item Genuine user participation
\item Firsthand experience with work practices
\item Anchoring visions
\end{itemize}
We tried to cover all of these, but found especially user participation and anchoring hard to fulfil.
User participation, specifically, I think we failed mostly because management was not very interested in the project.
Had they been more invested, we could have been able to get more time from the employees to discuss our understandings and ideas.
Taken from Participatory IT Design (Bødker et al., 2004) p. 59: "Management is responsible for allotting the time and information resources necessary for making user participation happen."
But we have not been very persistent, so some of the blame lies with us as well.

Another reason, and also a problem for anchoring visions, is that Valcon is an hour away with a car, and longer without.
This led to us only taking 2 trips there and relying on Michael, who works there, to include his colleagues in the project and anchor our visions.

My last thought on this is that there was not much two-way communication.
We expressed our ideas to Danni, the head of IT, and he understood, but the remaining employees involved in the process we only ever interviewed, never conversed with.
I do believe we explained our project to each of them at the beginning of the interviews, but there was never any follow-up.

\subsubsection{What could we have done differently to improve this?}
We could have upheld the principles during the entire project.
In fact, I had forgotten they existed until the last lectures of exam preparation, where the entire course was revisited.
If we had tried to uphold them through the entire project, we could have acted sooner on the issues mentioned above - we could have insisted on going a third time, chatted more with the employees about the project, or worked our way around the limitations set by the management a little better.

But in the end I believe we did the best we could.
The only thing I think would have actually helped here is more experience, and that is what this course is here for.