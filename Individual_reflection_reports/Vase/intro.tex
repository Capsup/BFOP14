\subsubsection{Introduction}
All in all, I think our project went well.
We got the problem early, blazed through the initiation and in-line phases, spent a good amount of time in-depth, and worked hard at identifying the best solutions.

In my experience, the most useful tools we were given were the business canvas, the MUST structure, and the ethnographic tools for interviewing and observing.

The business canvas helped us get a quick understanding of what was important to Valcon, and led us to understanding later additions and changes to it without problems.
I don't know that Business Model Generation (Osterwalder et al., 2010) specifically talks about this, but I believe it is one of the greatest strengths of the canvas: that even if I have misunderstood some part of the company, it's easy to find and correct the mistake and I quickly understand the correction.

We followed the MUST phases.
We decided early that we wouldn't be too bound by them, as the amount of time we had with the company was limited.
We started the in-line phase before the initiation phase was complete and the in-depth phase before the in-line phase was complete.
We thought of possible solutions from the start to the end.
This worked out very well, though we may have lost some of our overview because of it.
Generally, the phases gave us a clear path to follow, while allowing the flexibility of going quickly through less relevant areas.
I discuss the principles below.

The following sections are my analyses of some of the problems we faced during the project.