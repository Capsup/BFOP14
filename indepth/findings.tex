\subsection{Results from interviews}
There are several reasons why the current recruitment process needs to be optimized.

1. A lot of time is spent communicating information back and forth between Accounting and Recruitment.
(Appendix \quoteref{app:lisbeth}{lisbeth_vende_tilbage})
Each department maintains their own data on the new employees in individual spreadsheets.
(Appendix \quoteref{app:lisbeth}{lisbeth_ark})
Similarly time is spent communicating the information to the IT-department.
(Appendix \quoteref{app:lisbeth}{lisbeth_IT})
With a common system for keeping the information this could be avoided, as everyone could view the data within that system.

2. The main reason for frustration in the IT-department is that some recruitments occur with very short notice, and that data occasionally is incomplete.
(Appendix \todo{source})
The short notices are mainly rooted in recruitment of subcontractors in OMT,
while the incomplete data occurs both with recruitments in Valcon and OMT.
(Appendix \todo{source})
Part of the problem with incomplete data is that the initiators aren't all aware of which information the IT-department needs in order to set the new employee up correctly.
(Appendix \todo{source})

3. Time is spent on routine tasks, such as defining new initials, which could be automated if the data was kept in a standard format.
(Appendix \todo{source})

4. At Valcon, Accounting and IT support more standardized data, while Recruitment find it problematic because all data isn't necessarily available by the time of recruitment.
Furthermore, forcing standardized data might put additional pressure on the consultant/manager in charge of the recruitment, which should be avoided.
At OMT they also support more standardized data.

A visual representation of the chain of problems can be found in appendix \ref{app:ProblemChain}.
A transcript of the interviews can be found in appendix \ref{app:interviews}.

\subsection{The numbers}
\begin{wrapfigure}{r}{0.5\textwidth}
\vspace{-20pt}
\centering
\includegraphics[width=0.45\textwidth]{appendix/total_hires_per_half_year.png}
\label{fig:total_hires_per_half_year}
%\caption{Total number of new hires per half year.}
\end{wrapfigure}
Looking at the number of new hires in Valcon we got an impression of the scope of the problem.
Even though there are some fluctuations there is a trend toward a growth in the number of hires.
This corresponds with the expressed business strategy of growth.
However, the data is not entirely conclusive, as we have only been able to look at the number of new hires within the last 30 months.
A table with all the numbers of new hires can be found in appendix \ref{app:recruitment_data} together with additional graphs.
\begin{comment}
\begin{enumerate}
	\item IT-department
	\begin{enumerate}
		\item Has an interest in automating as many processes as possible
		\item Current process forces them to put their other work aside to make time for the employment process
	\end{enumerate}
	\item Recruitment-department
	\begin{enumerate}
		\item Wants to keep the flat structure that Valcon currently employs
		\item Wants to keep the recruitment process as transparent as possible for the consultants
		\item Doesn't want to put additional work onto the consultants nor management
	\end{enumerate}
	\item Accounting
	\begin{enumerate}
		\item Current process rests on a single person, creating a bottleneck and resulting in the process halting during vacation or illness. Makes it hard to transfer the work to another employee.
	\end{enumerate}
	
\end{enumerate}
\end{comment}