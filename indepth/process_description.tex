The process starts when a new employee has been hired.
An initiator (often Hanne or Jytte, sometimes other people) contacts accounting with information on the new employee.
A contract has to be formulated and, if there is information missing, the new employee has to be contacted in order to complete their particulars.

The information is entered into QHR and IT is contacted to let them know that a new employee has been hired.
IT then uses the information in QHR to generate employee initials, create a new user in the AD, and give them a mailbox on the Exchange Server, and then contacts accounting again with the employee's initials.
Accounting, after receiving the initials, enters the employee's particulars in Maconomy and Bluegarden (or in the case of foreign departments, the relevant salary system for that country).

At the same time, unless the employee is a subcontractor, IT contacts the new employee about their preferences for phone and internet.
When they have received these from the employee they contact Valcon's communications provider in order to set the new employee up according to their preferences.

Also at the same time, a PC is set up by IT according to the new employee's needs, based on which department they will work for.
Finally information about the handed out gear is recorded in TechAdm.

Occasionally, there is an urgent need for setting a new employee up faster than regularly, because they have been recruited on a very short notice. When this happens, the recruiter contacts IT before contacting accounting so that they can set up a computer and IT account faster.

Our understanding of the process is based on our interviews which can be found in appendix \ref{app:interviews}. Each interview describes a separate part of the process with overlaps.
A visual representation of the process can be found in appendix \ref{app:ProcessChart}. A list of each task and whether or not it can be automated can be found in \ref{app:automatise}.